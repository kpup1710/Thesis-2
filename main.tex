\documentclass[oneside, a4paper]{book}
\usepackage[left=3.5cm,right=2cm,top=3.5cm,bottom=3cm]{geometry}
\usepackage{indentfirst}
\usepackage{hyperref}
\usepackage{microtype}
\usepackage{enumitem}
\usepackage{times}
\usepackage{float}
\usepackage{graphicx}
\usepackage{wrapfig}
\usepackage{tikz, pgf} 
\usetikzlibrary{patterns}
\usepackage{float}
\usepackage{listings}
\usepackage{xcolor}
\usepackage[ruled, linesnumbered]{algorithm2e}
\newcommand{\R}{\mathbb{R}}
\definecolor{codegreen}{rgb}{0,0.6,0}
\definecolor{codegray}{rgb}{0.5,0.5,0.5}
\definecolor{codepurple}{rgb}{0.58,0,0.82}
\definecolor{backcolour}{rgb}{0.95,0.95,0.92}
\lstdefinestyle{mystyle}{
	backgroundcolor=\color{backcolour},   
	commentstyle=\color{codegreen},
	keywordstyle=\color{magenta},
	numberstyle=\tiny\color{codegray},
	stringstyle=\color{codepurple},
	basicstyle=\ttfamily\footnotesize,
	breakatwhitespace=false,         
	breaklines=true,                 
	captionpos=b,                    
	keepspaces=true,                 
	numbers=left,                    
	numbersep=5pt,                  
	showspaces=false,                
	showstringspaces=false,
	showtabs=false,                  
	tabsize=2
}
\hypersetup{colorlinks,linkcolor={black},citecolor={blue},urlcolor={red}}  \usepackage{multido}

\newcommand{\Pointilles}[1]{%
  \par\nobreak
  \noindent\rule{0pt}{1.5\baselineskip}% Provides a larger gap between the preceding paragraph and the dots
  \multido{}{#1}{\noindent\makebox[\linewidth]{\dotfill}\endgraf}% ... dotted lines ...
  \bigskip% Gap between dots and next paragraph
}
\lstset{style=mystyle}
\usepackage{amsmath,amssymb,latexsym,amscd,amsxtra,graphicx,graphpap,amsthm}
\usepackage{mathtools}
\usepackage{pb-diagram}
\usepackage{indentfirst}
\usepackage{caption}
\usepackage{aligned-overset}
\usepackage{picinpar,floatflt}
\usepackage{longtable}
\usepackage[font=small,labelfont=bf]{caption}
\usepackage{fancybox}
\usepackage[utf8]{vietnam}
\usepackage[tight,vietnam]{minitoc}
\usepackage{anyfontsize}
\usepackage{biblatex}
\newcommand{\bvec}[1]{\textbf{#1}}
\usepackage{multicol}
\usepackage{multirow}
%\usepackage{commath}
\usepackage{pgfplots}
\pgfplotsset{compat=1.15}
\usepackage{pst-plot}
\usepgfplotslibrary{fillbetween} 
%==================================
\usepackage{titletoc}
\titlecontents{chapter}
  [0pt]
  {\bfseries}
  {\chaptername\ \thecontentslabel\enskip}
  {}
  {\hfill\contentspage}

%==================================

%==================================
\renewcommand{\thechapter}{\arabic{chapter}}
\renewcommand{\thesection}{\arabic{chapter}.\arabic{section}}
\newcommand{\eproof}{\hfill $\square$}
%\newcommand{\eproof}{\hfill}
\newcommand{\chm}{{\bfseries Chứng minh.}}
\newenvironment{cm}{\chm}{\eproof}
%===============================
\theoremstyle{plain}
\newtheorem{bd}{Bổ đề}[chapter]
\providecommand*{\bdautorefname}{Bổ đề}
\newtheorem{md}{Mệnh đề}[chapter]
\providecommand*{\mdautorefname}{Mệnh đề}
\newtheorem{hq}{Hệ quả}[chapter]
\providecommand*{\hqautorefname}{Hệ quả}
\newtheorem{dl}{\bfseries Định lý}[chapter]
\providecommand*{\dlautorefname}{Định lý}
% \newtheorem{algorithm}{\bfseries Thuật toán}[chapter]
\providecommand*{\algorithmautorefname}{Thuật toán}
\newtheorem*{namedthm}{\namedthmname}
\newcounter{namedthm}
\makeatletter
\newenvironment{named}[1]
  {\def\namedthmname{#1}%
   \refstepcounter{namedthm}%
   \namedthm\def\@currentlabel{#1}}
  {\endnamedthm}
\makeatother

\theoremstyle{definition}
\newtheorem{dn}{Định nghĩa}[chapter]
\providecommand*{\dnautorefname}{Định nghĩa}
\newtheorem{bt}{Bài toán}[chapter]
\providecommand*{\btautorefname}{Bài toán}
\newtheorem{vd}{\bfseries Ví dụ}[chapter]
\providecommand*{\mdautorefname}{Mệnh đề}

\theoremstyle{remark}
\newtheorem{kh}{ký hiệu}[chapter]
\newtheorem{nx}{\bf Nhận xét}[chapter]
\newtheorem{cy}{Chú ý}[chapter]

\renewcommand{\baselinestretch}{1.5}
\setlength{\oddsidemargin}{0.3cm}     %Lề trái tính từ điểm cách mép giấy 2.54cm
\setlength{\topmargin}{-1cm}          %Lề trên tính từ điểm cách mép giấy 2.54cm
\setlength{\headsep}{0.5cm}                %Khoang cach tu  Headerline tới Khối chữ
\textwidth=15.5cm
\textheight=24.5cm
\renewcommand{\bibname}{Tài liệu tham khảo}
\renewcommand{\large}{\fontsize{14pt}{14pt}\selectfont}
\renewcommand{\Large}{\fontsize{16pt}{16pt}\selectfont}
\renewcommand{\LARGE}{\fontsize{15pt}{15pt}\selectfont}
\renewcommand{\th}{\fontsize{18pt}{18pt}\selectfont}
\makeatletter
\def\ps@myheadings{
\def\@evenhead{\hfil\thepage\hfil}
\def\@oddhead{
\hfil\thepage\hfil}}
\makeatother
\pagestyle{myheadings}

\usepackage{fancyhdr}

\fancyhf{}

% \fancyhead[C]{\thepage}

\pagestyle{fancy}

\fancypagestyle{plain}

% \fancyhf{} % clear all header and footer fields

\fancyfoot[C]{\bf \thepage}
\fancypagestyle{firstpage}{% define a custom header OPTION 1 to be applied to specific pages
  \fancyhf{}
    \fancyfoot[R]{\bf \ttfamily Trang \thepage}
% \fancyhead[L]{Trường Đại Học Bách Khoa Hà Nội}
\fancyfoot[L]{\bf \ttfamily Nhóm 1D}
}
\renewcommand{\headrulewidth}{0pt}
\newenvironment{mproof}{\paragraph{Chứng minh:}}{\hfill$\square$}
\newenvironment{myproof}[2] {\paragraph{Proof of {#1} {#2} :}}{\hfill$\square$}
\DeclarePairedDelimiter{\pro}{\langle}{\rangle}
\DeclarePairedDelimiter{\norm}{\lVert}{\rVert}
\DeclarePairedDelimiter{\abs}{\lvert}{\rvert}
\usepackage{xpatch}
\makeatletter
\AtBeginDocument{\xpatchcmd{\@thm}{\thm@headpunct{.}}{\thm@headpunct{}}{}{}}
\makeatother

\addbibresource{refs.bib}


\begin{document}
	
\thispagestyle{empty}
%\thispagestyle{headings}
\setcounter{page}{1}%
\pagenumbering{roman}
%\addcontentsline{toc}{chapter}{{Trang bìa phụ}}
%\newgeometry{top=2.0cm,bottom=3.0cm,left=3.5cm,right=2.8cm}
%\begin{titlepage}

%\setlength{\fboxrule}{1pt}
%\thisfancypage{\setlength{\fboxsep}{2pt}\setlength{\fboxrule}{2pt}\doublebox}{} 
%\setlength{\fboxrule}{1pt}
%\thisfancypage{\setlength{\fboxsep}{0pt}\setlength{\shadowsize}{0pt}\doublebox}{}
% \pagenumbering{gobble}
\newgeometry{left=2.8cm,right=1.8cm,top=2.5cm,bottom=2cm}
\begin{tikzpicture}[remember picture,overlay,inner sep=0,outer sep=0]
\draw[blue!70!black,line width=2.5pt] ([xshift=-1cm,yshift=-2cm]current page.north east) coordinate (A)--([xshift=2cm,yshift=-2cm]current page.north west) coordinate(B)--([xshift=2cm,yshift=2cm]current page.south west) coordinate (C)--([xshift=-1cm,yshift=2cm]current page.south east) coordinate(D)--cycle;
\end{tikzpicture}
\begin{center}
\Large
\vspace{-1cm}
{\bfseries TRƯỜNG ĐẠI HỌC BÁCH KHOA HÀ NỘI}\\
{\bfseries VIỆN TOÁN ỨNG DỤNG VÀ TIN HỌC}

\end{center}
\vspace{1cm}
\begin{center}
	\fontsize{17pt}{15pt}\selectfont
	\includegraphics[scale=0.06]{HUST_logo}
	\vspace{1cm}
	
	\fontsize{22pt}{15pt}\selectfont
	\bfseries Phương pháp hệ động lực giải một lớp bài toán\\
                    tối ưu hai cấp tựa lồi
\end{center}

\vspace{0.8cm}
\begin{center}
	\fontsize{16pt}{16pt}\selectfont 	
	{\bfseries ĐỒ ÁN II}\\	
	\fontsize{14pt}{16pt}\selectfont
	{\bfseries Chuyên ngành: TOÁN TIN}\\
	\fontsize{14pt}{16pt}\selectfont
	{\bfseries Chuyên sâu: Tối ưu}\\
\end{center}

\vspace{2cm}
\begin{center}
\fontsize{13pt}{16pt}\selectfont
\textbf{\begin{tabular}{l l}
Giảng viên hướng dẫn:& TS. Trần Ngọc Thăng\\
Sinh viên thực hiện:& Nguyễn Việt Dũng\\
Lớp:&{\bfseries Chương trình tài năng Toán tin K64}
\end{tabular}}

\end{center}
\vspace{2cm}
\begin{center}
\LARGE HÀ NỘI -- 2022
\end{center}


\restoregeometry


\newpage

\newpage
\newgeometry{left=2.2cm,right=2.8cm,top=2cm,bottom=2cm}
\pagenumbering{roman}
\setcounter{page}{1}
\thisfancypage{
	\setlength{\fboxsep}{0.5cm}
	\fbox}{}

\fontsize{13pt}{16pt}\selectfont
\bigbreak
\begin{center}
	{\bfseries NHẬN XÉT CỦA GIẢNG VIÊN HƯỚNG DẪN}
\end{center}
\bigbreak
% \fancyhf{}
% \thispagestyle{myheadings}
\fontsize{12pt}{14pt}\selectfont
\begin{enumerate}
	\item [{\bfseries 1.}]{\bfseries Mục tiêu và nội dung của đồ án:}
    \noindent\Pointilles{5}
	\item [{\bfseries 2.}] {\bfseries Kết quả đạt được:} 
	\noindent\Pointilles{5}
	\item [{\bfseries 3.}]{\bfseries Ý thức làm việc của sinh viên:}
	\noindent\Pointilles{5}
\end{enumerate}
\hspace{0.5\textwidth}
\begin{minipage}{0.5\textwidth}
	\noindent\begin{center}
		\textit{Hà Nội, ngày ... tháng ... năm 2022} \\
		Giảng viên hướng dẫn\\ \vspace{2cm}
		\textbf{TS. Trần Ngọc Thăng}
	\end{center}	
\end{minipage}
\restoregeometry



\newpage
\fontsize{22pt}{16pt}\selectfont
{\bfseries Lời cảm ơn}
\fontsize{13pt}{16pt}\selectfont
\bigskip

Để có thể hoàn thành đồ án này, em xin được gửi lời cảm ơn chân thành và
sâu sắc đến thầy TS. Trần Ngọc Thăng, thầy đã tận tình giảng dạy và hướng dẫn em trong suốt quá trình học tập và làm bài báo cáo.



\begin{minipage}{0.5\textwidth}
\end{minipage}
\hspace{0.5\textwidth}
\begin{minipage}{0.5\textwidth}
	\noindent\begin{center}
		\vspace{1cm}
		\textit{Hà Nội, tháng ... năm 2022} \\
		Tác giả đồ án\\ \vspace{1cm}
		\textbf{Nguyễn Việt Dũng}
	\end{center}	
\end{minipage}

\bigskip




\tableofcontents % Làm xuất hiện mục lục.
% \pagestyle{myheadings}


\newpage




%\addcontentsline{toc}{chapter}{Danh sách bảng}
% \thispagestyle{firstpage}
\pagenumbering{arabic}
\setcounter{page}{1}
%\fontsize{22pt}{16pt}\selectfont
%{\bfseries Bảng ký hiệu}
%\fontsize{13pt}{16pt}\selectfont
%\bigskip
% \addcontentsline{toc}{chapter}{\bfseries Bảng ký hiệu và chữ viết tắt}
%\thispagestyle{empty}

%\pagenumbering{arabic}
%\setcounter{page}{1}%
%%\pagenumbering{arabic}
%\addcontentsline{toc}{chapter}{Bảng ký hiệu}

% \begin{tabular}
% 	{@{\hspace{-0.1cm}} l @{\hspace{1.2cm}}p{11.5cm}l}
% 	$\mathbb R$ & tập các số thực\\
% 	$\mathbb R^n$ & không gian Euclid $n$ chiều\\
% 	$\nabla f(x)$ & véctơ gradient của hàm $f$ tại điểm $x$\\
% 	$P_C$& phép chiếu mêtric lên tập $C$\\
% 	VI& bài toán bất đẳng thức biến phân\\
% 	DVI & bài toán đối ngẫu với bài toán bất đẳng thức biến phân\\
% 	CP& bài toán bù\\
% 	LCP & bài toán bù tuyến tính
% \end{tabular}

%\newpage
%\addcontentsline{toc}{chapter}{\bfseries Danh sách bảng}
%\listoftables
%\thispagestyle{empty}

%\newpage
%\addcontentsline{toc}{chapter}{\bfseries Danh sách hình vẽ}
%\listoffigures

\chapter*{Mở đầu}
\large
\setlist{nosep}
Bài toán tối ưu hai cấp được thử nghiệm lần đầu tiên với mô hình toán học đề xuất bởi Bracken và McGill vào năm 1973 \cite{Bracken1973}, dựa trên công thức có dạng mô hình Stackelberg \cite{Stackelberg1934}. Sự khác biệt giữ lớp bài toán này là lớp bài toán tối ưu một cấp nằm ở cấu trúc phân tầng, ở đó ràng buộc của bài toán tối ưu này chịu tác động bởi một bài toán khác. Cụ thể, bài toán tối ưu hai cấp tổng quá được phát biểu như sau:
\begin{equation} \tag*{(GBP)}\label{GBP}
    \begin{aligned}
        & \min f(x, y) \quad \text { s.t. } \\
        & g_1(x, y) \leq 0, \quad x \in \mathbb{R}_{+}^{n_1}, \quad y \text { thuộc tập nghiệm của bài toán } \\
        R(x): & \min \left\{d(y) \mid g_2(C(x), y) \leq 0, \quad y \in \mathbb{R}_{+}^{n_2}\right\}
\end{aligned}
\end{equation}
trong đó, ánh xạ $C: \R^{n_1} \to \R^m$ và các hàm $f(x; y), d(y), g_1(x; y), g_2(u; y)$ liên tục, tập $D:= \{(x, y) \in \R^{n1}_+ \times \R^{n_2}_+\ |\ g_1(x, y) \leq 0, g_2(C(x), y) \leq 0  \}$ khác rỗng và compact.\\
\indent Không giống như bài toán tối ưu một cấp, khi ta có thể liên tưởng bài toán này với một tác tử đang cố gắng đạt được mục tiêu. Bài toán tối ưu hai cấp tương tự với việc hai tác tử cố gắng đạt được mục tiêu trong khi tác động qua lại lẫn nhau thông qua các quyết định của chúng trong quá khứ. Hơn nữa, phần lớn các bài toán trong thực tế chịu tác động bởi nhiều tác nhân, số lượng các nghiên cứu liên quan đến tối ưu hai cấp đã tăng đáng kể trong thập kỷ vừa qua và ứng dụng của chúng đã lan rộng đến nhiều lĩnh vực. Một số nghiên cứu có tính thực tiễn ví dụ như bài toán cân bằng Nash trong kinh tế \cite{Nash1950}, bài toán tối ưu điều khiển \cite{Kalashnikov2018}, bài toán thiết kế mạng lưới giao thông vận tải \cite{Ukkusuri2013,Angulo2014}, bài toán quản lý chuỗi cung ứng \cite{Gao2011,Lukac2008}, bài toán học máy \cite{Franceschi2018,MacKay2019},... \\
\indent Có hai cách để phân loại bài toán tối ưu hai cấp, ,một số nhà nghiên cứu phân loại theo lớp hàm, ví dụ như tuyến tính \cite{Belkhiri2021,Muu2000,Jorge2005}, lồi \cite{Bonnel2015,Horst1999}, không lồi \cite{Lin2014,Mitsos2008}, hoặc rời rạc \cite{Alekseeva2016,Calvete2011}. Các nhà nghiên cứu khác phân loại theo ngữ cảnh của bài toán, ví dụ bài toán optimistic và bài toán pessimistic, bài toán tối ưu trên tập nghiệp hữu hiệu, bài toán tối ưu đa mục tiêu và bài toán tối ưu hai cấp cùng với bài toán tối ưu hai cấp đa mục tiêu.\\
\indent Về khía cạnh phương pháp giải, các phương pháp giải chính xác với cách tiếp cận cổ điện đã được nghiên cứu qua hàng thập kỷ. Có thể kể đến như là phương pháp điểm tới hạn, phương pháp nhánh cận, phương pháp hướng giảm, phương pháp hàm phạt và phương pháp miền tin cậy. Ngoài ra ta còn có các phương pháp heuristic với thuật toán tiến hóa, thuật toán tìm kiếm lân cận và các thuật toán dựa trên mạng nơ-ron \cite{Lu2016}.\\
\indent Mục tiêu của đồ án này là nghiên cứu bài toán tối ưu hai cấp dưới dạng bài toán tối ưu trên tập nghiệm hữu hiệu. Cụ thể, bài toán cấp dưới được xét có hàm mục tiêu là hàm vector và bài cấp trên là hàm vô hướng. Chủ đề này đã được chú ý rộng rãi từ nhiều năm trước, với các hàm mục tiêu có dạng tuyến tính hoặc lồi ở cả hai bài toán cấp trên và cấp dưới \cite{Horst2007,Le2003,Ren2016}, trong khi đó các lớp hàm tổng quát hơn vẫn chưa được nghiên cứu sâu \cite{Mitsos2008}.Các hướng tiếp cận phổ biến bao gồm phương pháp hàm phạt với việc chuyển đổi bài toán về trường hợp tối ưu một cấp \cite{White1993,Le2003}, và thuật toán nhánh cận, với việc giải các bài toán con với ràng buộc nới lỏng \cite{Belkhiri2021,Horst1999,Muu2000}. Đồ án này tập trung vào việc giải bài toán tối ưu hai cấp với hàm mục tiêu giả lồi ở cả bài toán cấp trên và cấp dưới và đồng thời đề xuất một phương pháp nhánh cận dựa trên phương pháp của tiến sĩ Trần Ngọc Thăng \cite{Thang2020} nhưng với hướng tiếp cận hệ động lực.\\

Nội dung chính của đồ án gồm 3 chương cụ thể như sau:
\begin{itemize}
	\item \textbf{Chương 1: Một số khái niệm và kết quả cơ bản.} Chương này trình bày một số khái niệm cơ sở phục vụ cho việc xây dựng nội
	dung những chương tiếp theo. Đầu tiên, Mục 1.1 trình bày về hàm
	tựa lồi, hàm tựa lõm và giả lồi cùng nhiều ví dụ minh họa và tính chất quan trọng. Đây là lớp hàm bao quát khá rộng và vẫn giữ được tính chất một nghiệm cực tiểu địa phương cũng là nghiệm tối ưu toàn cục. Tiếp theo, Mục 1.2 tập trung và các lớp hàm không trơn và các khái niệm liên quan ví dụ như đạo hàm theo hướng tổng quát. Các khái niệm về tập chuẩn, tập chuẩn đảo, đa hộp, đa hộp đảo được trình bày ở Mục 1.3 cùng nhiều tính chất thú vị liên quan.
	\item \textbf{Chương 2: Bài toán quy hoạch giả lồi không trơn.} Chương này giới thiệu về bài toán quy hoạc giả lồi không trơn với hàm mục tiêu đúng như tên gọi của bài toán và các ràng buộc bất đẳng thức gồm các hàm tựa lồi khả vi liên tục. Đây là một lớp bài toán quan trọng và sẽ được giải bởi phương pháp hệ động lực đề xuất bởi Liu cùng các cộng sự \cite{Liu2021}. Mô hình của thuật toán cùng với các chứng minh về sự tồn tại nghiệm và sự hội tụ sẽ được trình bày chi tiết tại chương này.
	\item \textbf{Chương 3: Bài toán tối ưu hai cấp tựa lồi (QBP)}. Chương này nghiên cứu về bài toán tối ưu hai cấp tựa lồi (QBP) và đề xuất thuật toán dựa trên mối liên hệ với bài toán quy hoạch giả lồi không trơn cùng với thuật toán xấp xỉ trong bằng đa hộp đảo. Tính đúng đắn và sự hội tụ của thuật toán sẽ được trình bày chi tiết ở đây. Cuối cùng, một số ví dụ tính toán sẽ được đưa ra nhằm minh họa tính hiệu quả của thuật toán.
\end{itemize}

\indent Đồ án được hoàn thành dưới sự hướng dẫn của TS. Trần Ngọc Thăng
tại Viện Toán ứng dụng và Tin học, Trường Đại Học Bách Khoa Hà Nội.
Mặc dù đã nỗ lực hết sức, tuy nhiên trong khoảng thời gian cho phép,
việc nghiên cứu và trình bày không thể tránh khỏi nhầm lẫn thiếu sót,
tôi mong nhận được sự giúp đỡ đóng góp của thầy cô và bạn bè để đồ
án được hoàn thiện hơn.

\chapter{Một số khái niệm và kết quả cơ bản}
Chương này của đồ án trình bày các khái niệm quan trọng và một số kết
quả liên quan, nhằm phục vụ cho việc xây dựng các lý thuyết và thuật
toán ở những chương sau. Cụ thể, Mục 1.1 và 1.2 giới thiệu một số lớp
hàm sẽ được sử dụng trong đồ án và khái niệm liên quan. Các khái
niệm về tập chuẩn, tập chuẩn đảo, đa hộp, đa hộp đảo được trình bày ở
Mục 1.3.
\section{Hàm lồi mở rộng}
\begin{dn}
Cho hàm $f: S \rightarrow [-\infty, \infty]$ trên tập $S \subset \R ^n$ khác rỗng, hàm $f$ được gọi là lồi trên $S$ nếu và chỉ nếu với bất kỳ $x^1, x^2 \in S$ và $\lambda \in [0,1]$, ta có:
\begin{equation*}
    f(\lambda x^1 +(1-\lambda) x^2) \leq \lambda f(x^1) + (1 - \lambda) f(x^2)
\end{equation*}
Mặt khác, một hàm $f$ được gọi là lõm trên $S$ nếu $-f$ là lồi trên $S$.
\end{dn}
\begin{dn}
    Cho hàm $f: S \rightarrow [-\infty, \infty]$ trên tập $S \subset \R ^n$ khác rỗng, hàm $f$ được gọi là tựa lồi (quasiconvex) trên $S$ nếu và chỉ nếu với bất kỳ $x^1, x^2 \in S$ và $\lambda \in [0,1]$, ta có:
    \begin{equation*}
         f(\lambda x^1 + (1 - \lambda)x^2) < \max \{f(x^1), f(x^2)\}
     \end{equation*}
     hay một cách tương đương:
     \begin{equation*}
          f(x^1) \geq f(x^2) \Rightarrow f(x^1) \geq f(x^1 + \lambda(x^2 - x^1))
     \end{equation*}
\end{dn}

\begin{dn}
    Với mỗi số thực $\alpha \in \R$, ta gọi
    $$ L_\alpha(h) := \{x \in S\ |\ f(x) \leq \alpha \} $$
    là tập mức dưới của hàm $f$ và
    $$ L^\alpha(h) := \{x \in S\ |\ f(x) \geq \alpha \} $$
    là tập mức trên của hàm $f$.
\end{dn}

\begin{dl}[\cite{gen_convex}]
    \label{quasiconvex_theo}
    Hàm $f$ xác định trên tập lồi $S \subseteq \R^n$ khác rỗng là hàm tựa lồi khi và chỉ khi tập mức dưới $L_\alpha$ của hàm đó lồi với mọi $\alpha \in \R$
\end{dl}
\begin{vd}
    Xét hàm số
    \begin{equation*}
        h(x) = \sqrt{|x|}
    \end{equation*}
    trên tập $\R$
\end{vd}
\begin{figure}
    \centering
    \begin{tikzpicture}[declare function={f(\x)=sqrt{abs(\x};}]
    \begin{axis}[
            width=0.8\textwidth,
            height=0.6\textwidth,
            axis on top,
            legend pos=outer north east,
            axis lines = center,
            % xticklabel style = {font=\tiny},
            % yticklabel style = {font=\tiny},
            xlabel = $x$,
            ylabel = $y$,
            legend style={cells={align=left}},
            legend cell align={left},
        ]
        \addplot[very thick,red,samples=161,domain=-3:3,name path=f] {f(x)};
        \addplot[very thick,red,samples=161, domain=-2:2, name path=g]
        {sqrt(2)};
        \addplot[red!10, opacity=0.4] fill between[of=f and g, soft clip={domain=-2:2}];
    \end{axis}
\end{tikzpicture}
    \caption{: Đồ thị của hàm số $h(x) = \sqrt{|x|}$ trên tập số thực $\R$}
    \label{fig:vd1.1}
\end{figure}
Dễ thấy mọi tập mức dưới $L_\alpha(h)$ của $h(x)$ đều là tập lồi (xem Hình
\ref{fig:vd1.1}) nên $h(x)$ là hàm tựa lồi trên $\R$.

\begin{vd} \label{vd:1.2}
    Xét hàm số $$h(x;y) = -xy$$ xác định trên $S = \{(x,y) \in \R^2 | x, y \geq 0 \}$
\end{vd}

\begin{figure}
    \centering
    \begin{tikzpicture}
        \begin{axis}[
            %hide axis,
            width=0.8\textwidth,
            height=0.6\textwidth,
            axis on top,
            axis lines = box,
            xlabel=$x$,ylabel=$y$,
            y dir = reverse,
                    mesh/interior colormap name=hot, colormap/blackwhite,
                    ]
            \addplot3[domain=0:10,surf,samples=41] {-x*y};
        \end{axis}
    \end{tikzpicture}
    \caption{: Đồ thị của hàm số $ h(x, y) = -xy$ trên $[0, 10]^2$}
    \label{fig:vd1.2}
\end{figure}

Với $\alpha$ bất kỳ thuộc $\R$ ta có tập mức dưới của $h$
\begin{equation*}
    L_\alpha(h) = \{(x, y) \in S | -xy \leq \alpha \} = \{(x, y) \in \R^2| x, y \geq 0, xy \geq -\alpha\}
\end{equation*}

\begin{vd}
    Xét hàm số
    \begin{equation*}
        h(x, y) = xy + x^2y^2 + x^3y^3
    \end{equation*}
    xác định trên $\R^2_+$
\end{vd}
\begin{figure}[h!]
    \centering
    \begin{tikzpicture}
        \begin{axis}[
            %hide axis,
            width=0.8\textwidth,
            height=0.6\textwidth,
            axis on top,
            axis lines = box,
            xlabel=$x$,ylabel=$y$,
            y dir = reverse,
                    mesh/interior colormap name=hot, colormap/blackwhite,
                    ]
            \addplot3[domain=0:10,surf,samples=41] {x*y + x^2*y^2 + x^3*y*3};
        \end{axis}
    \end{tikzpicture}
    \caption{: Đồ thị của hàm số $ h(x, y) = xy + x^2y^2 + x^3y^3$ trên $[0, 10]^2$}
    \label{fig:vd1.3}
\end{figure}
Đặt $g(t) = t + t^2 + t^3$ và $u(x, y) = xy$ thì $ h(x, y) = g(u(x, y))$. Từ Ví dụ \ref{vd:1.2}, ta có $-xy$ là hàm tựa lồi nên $u(x, y) = xy$ là hàm tựa
lõm. Do $g$ là hàm không giảm và $u$ là hàm tựa lõm trên $S$ nên $h$ là một
hàm tựa lõm trên $S$ (xem [2], tr. 57).

\begin{dn}
     Cho hàm $f: S \rightarrow [-\infty, \infty]$ trên tập $S \subset \R ^n$ khác rỗng, hàm $f$ được gọi là tựa lồi chặt (strictly quasiconvex) nếu và chỉ nếu với bất kỳ $x^1, x^2 \in S$, $x^1 \neq x^2$ và $\lambda \in [0,1]$, ta có:
     \begin{equation*}
         f(\lambda x^1 + (1 - \lambda)x^2) < \max \{f(x^1), f(x^2)\}
     \end{equation*}
     hay một cách tương đương:
     \begin{equation*}
          f(x^1) \geq f(x^2) \Rightarrow f(x^1) \geq f(x^1 + \lambda(x^2 - x^1))
     \end{equation*}
\end{dn}

\begin{vd}
Hàm 
$
f(x) = 
\begin{cases}
\dfrac{|x|}{x}, &x \neq 0 \\
0, & x = 0
\end{cases}
$
là hàm tựa lồi nhưng không tựa lồi chặt.
\end{vd}
Ta biết rằng điểm cực trị của một hàm lồi đồng thời là cực tiểu toán cục của hàm đó. Tuy nhiên, tính chất này không chỉ đúng với các hàm tựa lồi và tựa lồi chặt, vì vậy một lớp hàm mới được giới thiệu:
\begin{dn}[xem \cite{gen_convex}]
    \label{psuedoconvex_def}
    Cho một hàm $f$ khả vi trên một tập lồi mở $S \subset \R^n$, hàm $f$ được gọi là giả lồi trên $S$ khi và chỉ khi:
    \begin{equation}
    \label{dn1}
    x^1, x^2 \in S, f(x^1) > f(x^2) \Rightarrow \nabla f(x^1)^T (x^2 - x^1) < 0
    \end{equation}
\end{dn}
Nếu bất đẳng thức trong vế phải của \eqref{dn1} vẫn đúng trong trường hợp $f(x^1) = f(x^2)$ thì $f$ được gọi là giả lồi chặt (strictly psuedoconvex), hay phát biểu một cách đầy đủ như sau:
\begin{dn}
    Một hàm $f$ khả vi trên một tập lồi mở $S \subset \R^n$ được gọi là giả lồi chặt (strictly psuediconvex) trên $S$ khi và chỉ khi:
    \begin{equation*}
        x^1, x^2 \in S,\ x^1 \neq x^2,\ f(x^1) > f(x^2) \Rightarrow \nabla f(x^1)^T (x^2 - x^1) < 0
    \end{equation*}
\end{dn}

\begin{md}[\cite{Clarke1983}]
    \label{normal_cone_prop}
    Giả sử tập khác rỗng $\Omega \subset \R^n$ lồi và đóng. Ta định nghĩa nón pháp tuyến của $\Omega$ tại $x \in \Omega$ là tập các vector thỏa mãn,
    \begin{equation*}
        N_\Omega(x) = \left\{ v \in \R^n: v^T(x - x^\prime) \geq 0,\ \forall x^\prime \in \Omega \right\}
    \end{equation*}
    Nếu $G: \R^n \to \R$ liên tục Lipschitz tại lân cận $x$ và đại cực tiểu tại $x$, thì $0 \in N_\Omega(x) + \partial G(x)$
\end{md}

\begin{md}[\cite{gen_convex}]
    \label{md_quasi}
    Nếu $G: \R^n \to \R$ là một hàm khả vi, $G$ là hàm tựa lồi khi và chỉ khi với bất kỳ $x^1, x^2 \in \R^n$
    \begin{equation*}
        G(x^1) \leq G(x^2) \Rightarrow \nabla G(x)^T (x^1 - x^2) \leq 0
    \end{equation*}
\end{md}

\section{Hàm không trơn}
Tính chất lồi đóng một vai trò quan trọng trong tối ưu toán học. Đặc biệt, tính lồi của hàm là khái niệm quan trọng nhất trong việc xây dựng các điều kiện tối ưu. Trong lý thuyết tối ưu trơn (khi và đạo hàm là liên tục), tính khả vi đòi hỏi việc xấp xỉ tuyến tính một hàm bởi gradient, kéo theo việc xấp xỉ dưới một hàm lồi. Điều này có thể được mở rộng cho các hàm lồi không trơn dẫn đến khái niện \textit{dưới-gradient} và \textit{dưới-vi phân}. Một dưới-gradient bảo toàn các tính chất của gradient, cung cấp một xấp xỉ dưới cho một hàm số, chỉ có điều xấp xỉ này không phải là duy nhất trong trường hợp hàm không trơn. Vì lẽ đó, thay vì chỉ có duy nhất một vector gradient, ta có môt tập các  dưới-gradient gọi là dưới-vi phân.

\begin{dn} [Clarke \cite{Clarke1983}] \label{daohamtongquat}
Cho $ f: \R^n \rightarrow \R $ là một hàm liên tục Lipschitz tại $x \in \R^n$. Đạo hàm tổng quát theo hướng $d$ của $f$ tại $x$ được định nghĩa như sau:
\begin{equation*}
    f^o(x, d) = \limsup _{y \rightarrow x, t \downarrow 0} \frac{f(x+t d)-f(x)}{t} \text {. }
\end{equation*}
\end{dn} 

\begin{dn} [Clarke \cite{Clarke1983}] \label{daohamtheohuong}
    Cho $ f: \R^n \rightarrow \R $ là một hàm liên tục Lipschitz tại $x \in \R^n$. Đạo hàm theo hướng $d$ của $f$ tại $x$ được định nghĩa như sau:
    \begin{equation*}
        f^{\prime}(x ; v)=\lim _{t \downarrow 0} \frac{f(x+t d)-f(x)}{t} \text {, }
    \end{equation*}
\end{dn}

\begin{dn} [Clarke \cite{Clarke1983}]
    Cho $ f: \R^n \rightarrow \R $ là một hàm liên tục Lipschitz tại $x \in \R^n$. Dưới-vi phân của $f$ tại $x$ là tập $\partial f(x)$ các vector $\xi \in \R^n$ thỏa mãn:
    \begin{equation*}
        \partial f(x)=\left\{\xi \in \mathbb{R}^n: f^o(x ; v) \geq \xi^{\mathrm{T}} v\right., \forall v \in \left.\mathbb{R}^n\right\}
    \end{equation*}
\end{dn}

Dựa trên dưới-gradient và dưới-vi phân, ta có cách định nghĩa khác về hàm lồi như sau,
\begin{dn} [Penot \& Quang, 1997 \cite{Pen1997}]
    Cho $S \subset \R^n$ là một tập lồi khác rỗng. Hàm $f: S \rightarrow \R$ được gọi là giả lồi trên $K$, khi và chỉ khi với mọi $x^1, x^2, x^1 \neq x^2$, ta có
    \begin{equation*}
        \exists \eta \in \partial f(x): \eta^{\mathrm{T}}\left(x^2-x^1\right) \geq 0 \Rightarrow f\left(x^2\right) \geq f(x^1)
    \end{equation*}
\end{dn}
\begin{dl} 
Cho hàm $f: \R^n \rightarrow \R$ liên tục Lipschitz và khả vi tại $x \in \R^n$. Ta có,
\begin{equation*}
    \nabla f(x) \in \partial f(x)
\end{equation*}
\end{dl}

\begin{vd} [Hàm khả vi nhưng không trơn]
    Ta sẽ chứng minh hàm số sau: 
    \begin{equation}
    \label{eq:vd1.5}
    f(x) = 
        \begin{cases}
        0, &x = 0\\
        x^2\cos(\dfrac{1}{x}), &x \neq 0
        \end{cases}
    \end{equation}
    liên tục Lipscitz, khả vi tại mọi điểm nhưng không trơn (không khả vi liên tục).
    \begin{cm}
    Đầu tiên ta chỉ ra tính khả vi. Xét hàm $g(x) = x^2\cos(\dfrac{1}{x})$ khả vi tại mọi $x$ khác 0 và có đạo hàm:
    \begin{equation}
    \label{eq:f_prime}
        g^\prime(x) = \sin(\dfrac{1}{x}) + 2x\cos(\dfrac{1}{x})
    \end{equation}
    Do $g^\prime (x)$ liên tục khi $x \neq 0$, nên $f$ khả vi liên tục tại mọi $x \neq 0$. Mặt khác, 
\begin{equation}
    f(0+x) -f(0) = x^2\cos(\dfrac{1}{x})
\end{equation}
và $\lim_{x \to 0} |x| \cos(\dfrac{1}{x}) = 0$, suy ra hàm $f$ khả vi tại $x =0$ và $f^\prime(0) = 0$. Tuy nhiên, từ \eqref{eq:f_prime}, ta có giới hạn $\lim_{x \to 0} f^\prime(x)$ không tồn tại kéo theo f không khả vi liên tục.\\
\indent Tiếp theo ta chứng minh $f$ liên tục Lipschitz. Hiển nhiên $f$ liên tục Lipschitz tại $x \neq 0$, ta sẽ chứng minh $f$ liên tục Lipschitz tại $x=0$. Xét $-1 < y < z < 0$, ta có
\begin{align*}
|f(z)-f(y)|= & \left|\int_y^z f^{\prime}(x) d x\right| \leq \int_y^z \max _{x \in[y, z]}\left\{\left|f^{\prime}(x)\right|\right\} d x \\
= & \max _{x \in[y, z]}\left\{\left|\sin \left(\frac{1}{x}\right)+2 x \cos \left(\frac{1}{x}\right)\right|\right\}(z-y) \\
& \leq(1+2 \cdot 1 \cdot 1)|z-y|=3|z-y|,
\end{align*}
do đó thỏa mãn điều kiện Lipschitz. Do tính đối xứng của hàm $f$, điều kiện Lipschitz vẫn được thỏa mãn trong trường hợp $0 < y<z <1$.\\
\indent Bây giờ, xét trường hợp $-1 < y < 0 $ và $0 < z < 1$, ta có $n |y + z| < |y - z| $ và $f (-z) = f (z)$, vì vậy,
\begin{equation*}
     |f (y) - f (z)| = | f (y) - f (-z)| \leq 3 |y + z| \leq 3 |y - z|
\end{equation*}
\indent Cuối cùng, xét $y = 0$ và $z \in (-1, 1) \ \{0\}$. Ta có,
\begin{equation*}
    |f(0)-f(z)|=\left|z^2 \cos \left(\frac{1}{z}\right)\right| \leq|z| 1 \cdot 1=|0-z|
\end{equation*}
và ta được điều phải chứng minh.
\end{cm}
\end{vd}
\begin{figure}
    \centering
    \begin{tikzpicture}
    \begin{axis}[
            width=0.8\textwidth,
            height=0.6\textwidth,
            axis on top,
            legend pos=outer north east,
            axis lines = center,
            % xticklabel style = {font=\tiny},
            % yticklabel style = {font=\tiny},
            xlabel = $x$,
            ylabel = $y$,
            legend style={cells={align=left}},
            legend cell align={left},
        ]
        \addplot[very thick,red,samples=161,domain=0.001:0.2,name path=f] {x^2*cos(180/(x*pi))};
        \addplot[very thick,red,samples=161,domain=-0.2:0.001,name path=f] {x^2*cos(180/(x*pi))};
    \end{axis}
\end{tikzpicture}
    \caption{: Đồ thị của hàm số\eqref{eq:vd1.5} trên đoạn $[-0.2, 0.2]$}
    \label{fig:vd1.5}
\end{figure}

\begin{vd} Xét hàm giá trị tuyệt đối của $x$ trên tập số thực
$$
f(x)=|x| .
$$
Dưới-vi phân của hàm này tại  $x=0$ được cho bởi
$$
\partial f(0)=\operatorname{conv}\{-1,1\}=[-1,1] .
$$
\end{vd}

\begin{md}[\cite{Clarke1983}]
    \label{convex_compact}
    Cho hàm $G: \R^n \to \R$ liên tục Lipschitz tại lân cận $x^0 \in \R^n$, ta có $\partial G(x)$ nửa liên tục trên tại $x^0$ và là một tập con khác rỗng lồi và compact của $\R^n$.
\end{md}

% \begin{cm} 
% Theo định nghĩa \ref{daohamtheohuong}, $f^\prime (x; v)$ tồn tại với mọi $v \in \R^n$ và $f^\prime = \nabla f(x)^T v$
% \end{cm}

\begin{bd}[\cite{Clarke1983}]\label{lem:2.7}
 Giả sử $\left\{f_i, i=1,2,3, \ldots, n\right\}$ là tập hữu hạn các hàm liên tục Lipschitz tại lân cận $x$ và là hàm chính quy.
    Đặt
    $$
    \psi(x)=\max \left\{f_i(x), i=1,2,3, \ldots, n\right\}
    $$
    vậy
    $\partial \psi(x)=\operatorname{conv}\left\{\partial f_i(x), i \in I(x)\right\}$
    trong đó $I(x)$ là tập chỉ số $i$ thỏa mãn $f_i(x)=\psi(x)$.
    \end{bd}
% \section{Hàm không trơn}
Tính chất lồi đóng một vai trò quan trọng trong tối ưu toán học. Đặc biệt, tính lồi của hàm là khái niệm quan trọng nhất trong việc xây dựng các điều kiện tối ưu. Trong lý thuyết tối ưu trơn (khi và đạo hàm là liên tục), tính khả vi đòi hỏi việc xấp xỉ tuyến tính một hàm bởi gradient, kéo theo việc xấp xỉ dưới một hàm lồi. Điều này có thể được mở rộng cho các hàm lồi không trơn dẫn đến khái niện \textit{dưới-gradient} và \textit{dưới-vi phân}. Một dưới-gradient bảo toàn các tính chất của gradient, cung cấp một xấp xỉ dưới cho một hàm số, chỉ có điều xấp xỉ này không phải là duy nhất trong trường hợp hàm không trơn. Vì lẽ đó, thay vì chỉ có duy nhất một vector gradient, ta có môt tập các  dưới-gradient gọi là dưới-vi phân.

\begin{dn} [Clarke \cite{Clarke1983}] \label{daohamtongquat}
Cho $ f: \R^n \rightarrow \R $ là một hàm liên tục Lipschitz tại $x \in \R^n$. Đạo hàm tổng quát theo hướng $d$ của $f$ tại $x$ được định nghĩa như sau:
\begin{equation*}
    f^o(x, d) = \limsup _{y \rightarrow x, t \downarrow 0} \frac{f(x+t d)-f(x)}{t} \text {. }
\end{equation*}
\end{dn} 

\begin{dn} [Clarke \cite{Clarke1983}] \label{daohamtheohuong}
    Cho $ f: \R^n \rightarrow \R $ là một hàm liên tục Lipschitz tại $x \in \R^n$. Đạo hàm theo hướng $d$ của $f$ tại $x$ được định nghĩa như sau:
    \begin{equation*}
        f^{\prime}(x ; v)=\lim _{t \downarrow 0} \frac{f(x+t d)-f(x)}{t} \text {, }
    \end{equation*}
\end{dn}

\begin{dn} [Clarke \cite{Clarke1983}]
    Cho $ f: \R^n \rightarrow \R $ là một hàm liên tục Lipschitz tại $x \in \R^n$. Dưới-vi phân của $f$ tại $x$ là tập $\partial f(x)$ các vector $\xi \in \R^n$ thỏa mãn:
    \begin{equation*}
        \partial f(x)=\left\{\xi \in \mathbb{R}^n: f^o(x ; v) \geq \xi^{\mathrm{T}} v\right., \forall v \in \left.\mathbb{R}^n\right\}
    \end{equation*}
\end{dn}

Dựa trên dưới-gradient và dưới-vi phân, ta có cách định nghĩa khác về hàm lồi như sau,
\begin{dn} [Penot \& Quang, 1997 \cite{Pen1997}]
    Cho $S \subset \R^n$ là một tập lồi khác rỗng. Hàm $f: S \rightarrow \R$ được gọi là giả lồi trên $K$, khi và chỉ khi với mọi $x^1, x^2, x^1 \neq x^2$, ta có
    \begin{equation*}
        \exists \eta \in \partial f(x): \eta^{\mathrm{T}}\left(x^2-x^1\right) \geq 0 \Rightarrow f\left(x^2\right) \geq f(x^1)
    \end{equation*}
\end{dn}
\begin{dl} 
Cho hàm $f: \R^n \rightarrow \R$ liên tục Lipschitz và khả vi tại $x \in \R^n$. Ta có,
\begin{equation*}
    \nabla f(x) \in \partial f(x)
\end{equation*}
\end{dl}

\begin{vd} [Hàm khả vi nhưng không trơn]
    Ta sẽ chứng minh hàm số sau: 
    \begin{equation}
    \label{eq:vd1.5}
    f(x) = 
        \begin{cases}
        0, &x = 0\\
        x^2\cos(\dfrac{1}{x}), &x \neq 0
        \end{cases}
    \end{equation}
    liên tục Lipscitz, khả vi tại mọi điểm nhưng không trơn (không khả vi liên tục).
    \begin{cm}
    Đầu tiên ta chỉ ra tính khả vi. Xét hàm $g(x) = x^2\cos(\dfrac{1}{x})$ khả vi tại mọi $x$ khác 0 và có đạo hàm:
    \begin{equation}
    \label{eq:f_prime}
        g^\prime(x) = \sin(\dfrac{1}{x}) + 2x\cos(\dfrac{1}{x})
    \end{equation}
    Do $g^\prime (x)$ liên tục khi $x \neq 0$, nên $f$ khả vi liên tục tại mọi $x \neq 0$. Mặt khác, 
\begin{equation}
    f(0+x) -f(0) = x^2\cos(\dfrac{1}{x})
\end{equation}
và $\lim_{x \to 0} |x| \cos(\dfrac{1}{x}) = 0$, suy ra hàm $f$ khả vi tại $x =0$ và $f^\prime(0) = 0$. Tuy nhiên, từ \eqref{eq:f_prime}, ta có giới hạn $\lim_{x \to 0} f^\prime(x)$ không tồn tại kéo theo f không khả vi liên tục.\\
\indent Tiếp theo ta chứng minh $f$ liên tục Lipschitz. Hiển nhiên $f$ liên tục Lipschitz tại $x \neq 0$, ta sẽ chứng minh $f$ liên tục Lipschitz tại $x=0$. Xét $-1 < y < z < 0$, ta có
\begin{align*}
|f(z)-f(y)|= & \left|\int_y^z f^{\prime}(x) d x\right| \leq \int_y^z \max _{x \in[y, z]}\left\{\left|f^{\prime}(x)\right|\right\} d x \\
= & \max _{x \in[y, z]}\left\{\left|\sin \left(\frac{1}{x}\right)+2 x \cos \left(\frac{1}{x}\right)\right|\right\}(z-y) \\
& \leq(1+2 \cdot 1 \cdot 1)|z-y|=3|z-y|,
\end{align*}
do đó thỏa mãn điều kiện Lipschitz. Do tính đối xứng của hàm $f$, điều kiện Lipschitz vẫn được thỏa mãn trong trường hợp $0 < y<z <1$.\\
\indent Bây giờ, xét trường hợp $-1 < y < 0 $ và $0 < z < 1$, ta có $n |y + z| < |y - z| $ và $f (-z) = f (z)$, vì vậy,
\begin{equation*}
     |f (y) - f (z)| = | f (y) - f (-z)| \leq 3 |y + z| \leq 3 |y - z|
\end{equation*}
\indent Cuối cùng, xét $y = 0$ và $z \in (-1, 1) \ \{0\}$. Ta có,
\begin{equation*}
    |f(0)-f(z)|=\left|z^2 \cos \left(\frac{1}{z}\right)\right| \leq|z| 1 \cdot 1=|0-z|
\end{equation*}
và ta được điều phải chứng minh.
\end{cm}
\end{vd}
\begin{figure}
    \centering
    \begin{tikzpicture}
    \begin{axis}[
            width=0.8\textwidth,
            height=0.6\textwidth,
            axis on top,
            legend pos=outer north east,
            axis lines = center,
            % xticklabel style = {font=\tiny},
            % yticklabel style = {font=\tiny},
            xlabel = $x$,
            ylabel = $y$,
            legend style={cells={align=left}},
            legend cell align={left},
        ]
        \addplot[very thick,red,samples=161,domain=0.001:0.2,name path=f] {x^2*cos(180/(x*pi))};
        \addplot[very thick,red,samples=161,domain=-0.2:0.001,name path=f] {x^2*cos(180/(x*pi))};
    \end{axis}
\end{tikzpicture}
    \caption{: Đồ thị của hàm số\eqref{eq:vd1.5} trên đoạn $[-0.2, 0.2]$}
    \label{fig:vd1.5}
\end{figure}

\begin{vd} Xét hàm giá trị tuyệt đối của $x$ trên tập số thực
$$
f(x)=|x| .
$$
Dưới-vi phân của hàm này tại  $x=0$ được cho bởi
$$
\partial f(0)=\operatorname{conv}\{-1,1\}=[-1,1] .
$$
\end{vd}

\begin{md}[\cite{Clarke1983}]
    \label{convex_compact}
    Cho hàm $G: \R^n \to \R$ liên tục Lipschitz tại lân cận $x^0 \in \R^n$, ta có $\partial G(x)$ nửa liên tục trên tại $x^0$ và là một tập con khác rỗng lồi và compact của $\R^n$.
\end{md}

% \begin{cm} 
% Theo định nghĩa \ref{daohamtheohuong}, $f^\prime (x; v)$ tồn tại với mọi $v \in \R^n$ và $f^\prime = \nabla f(x)^T v$
% \end{cm}

\begin{bd}[\cite{Clarke1983}]\label{lem:2.7}
 Giả sử $\left\{f_i, i=1,2,3, \ldots, n\right\}$ là tập hữu hạn các hàm liên tục Lipschitz tại lân cận $x$ và là hàm chính quy.
    Đặt
    $$
    \psi(x)=\max \left\{f_i(x), i=1,2,3, \ldots, n\right\}
    $$
    vậy
    $\partial \psi(x)=\operatorname{conv}\left\{\partial f_i(x), i \in I(x)\right\}$
    trong đó $I(x)$ là tập chỉ số $i$ thỏa mãn $f_i(x)=\psi(x)$.
    \end{bd}
\section{Tập chuẩn và đa hộp}
\begin{dn} [Tuy2000 \cite{Tuy2000}]
Một tập $Q \subset \R^p_+$ được gọi là một tập chuẩn (normal set) nếu với mọi $q \in \R^p_+$, ta có
\begin{equation*}
    q \in Q \Rightarrow (q - \R^p_+) \cap \R^p_+ \subseteq Q \text{.}
\end{equation*}
\end{dn}

\begin{vd}
Từ định nghĩa trên ta thấy $\empty, \{0\}, \R^p_+$ đều là các tập chuẩn. Hình \ref{fig:1} là ví dụ cho một tập chuẩn trong không gian hai chiều.
\end{vd}

\begin{md}[Tuy2000 \cite{Tuy2000}]
     Nếu $Q$ là một tập chuẩn thì $ Q \cup \{q \in R^p_+ | q_i = 0, i \in \{1, . . . , p\}\}$ cũng là một tập chuẩn.

\end{md}

\begin{dn} [Tuy2000 \cite{Tuy2000}]
    Một tập $Q \subset \R^p_+$ được gọi là một \textit{tập chuẩn đảo (reversed normal set)} nếu với mọi $q \in \R^p_+$, ta có
    \begin{equation*}
        q \in Q \Rightarrow q + \R^p_+ \subseteq Q \text{.}
    \end{equation*}
\end{dn}
Cho $0 \leq d, Q \subseteq [0, d]$ được gọi là một tập chuẩn đảo trong hộp $[0, d]$ nếu với mọi $q \in [0, d]$, ta có
\begin{equation*}
    q \in Q \Rightarrow (q + \R^p_+) \cap (d - \R^p_+) \subseteq Q.
\end{equation*}

\begin{figure}[!h]
    \centering
    \begin{tikzpicture}[
        > = stealth, 
        line width = 0.3mm, 
        scale = 0.7
    ]
        \draw[fill = pink, fill opacity = 0.5, draw = none] (0,10) -- (1,9) -- (4,9) -- (9,5) -- (10,1) -- (11,0) -- (0,0) -- cycle;
        \draw[->] (-1,0) -- (13,0);
        \draw[->] (0,-1) -- (0,13);
        \draw[line width = 0.5mm] (0,10) -- (1,9) -- (4,9) -- (9,5) -- (10,1) -- (11,0);
    \end{tikzpicture}
    
    \caption{Một tập chuẩn trong trường hợp hai chiều (được minh họa bằng phần màu hồng).}
    \label{fig:1}
\end{figure}

\begin{md} [Tuy2000 \cite{Tuy2000}]
    \label{union_iter_normalset}
    Cho các tập chuẩn (t.ư., tập chuẩn
đảo) $ Q_1, Q_2, \dots , Q_m$. Khi đó, $Q1 \cup Q2 \cup \dots \cup Q_m$ và $Q_1 \cap Q_2 \cap \dots \cap Q_m$ cũng là các tập chuẩn (t.ư., tập chuẩn đảo)
\end{md}

\begin{dn} [Tuy1999 \cite{Tuy1999}]
    Cho một tập $Q \in \R^p$, ta gọi tập
    \begin{align*}
        N(Q) &:= (Q - \R^p_+) \cap \R^p\\
        \text{t.ư, $N^r_d(Q)$} &= (Q + \R^p_+) \cap (d - \R^p_+)
    \end{align*}
là \textit{bao chuẩn (normal hull)} của $Q$ (t.ư., \textit{bao chuẩn đảo (reverse normal hull)} của tập chuẩn đảo $Q$ trong hộp $[0, d]$.
\end{dn}
Khi đó, $N^r_d(Q)$ là tập chuẩn đảo nhỏ nhất chứa $Q$ trong hộp $[0, d]$.

\begin{dn} [Tuy1999 \cite{Tuy1999}]
    Một điểm $q \in \R^p$ được gọi là \textit{điểm cực biên dưới (lower extreme point)} của tập chuẩn đảo compact $Q \subset \R^p_+$ nếu với mỗi $q^\prime \in Q$
    \begin{equation*}
        q^\prime \leq q \Rightarrow q^\prime= q \text{.}
    \end{equation*}
\end{dn}
Tập tất cả các điểm biên cực dưới của $Q$ được ký hiệu là $EX(Q)$.
\begin{dl} [Tuy1999 \cite{Tuy1999}]
Cho $f(x)$ là một hàm tăng trên tập compact $Q$. Khi đó, giá trị cực tiểu của hàm $f(x)$ trên $Q$ bằng giá trị cực tiểu của nó trên $N^r_d(Q)$.  Hơn nữa, giá trị cực đại đó đạt tại một điểm cực biên dưới của $N^r_d(Q)$.
\end{dl}

\begin{figure}[!h]
    \centering
    \begin{tikzpicture}[
        > = stealth, 
        line width = 0.3mm, 
        scale = 0.7
    ]
        \draw[fill = pink, fill opacity = 0.5, draw = none] (0,10) -- (1,5) -- (4,4) -- (5,2) -- (10,1) -- (11,0) -- (11,10) -- cycle;
        \draw[->] (-1,0) -- (13,0);
        \draw[->] (0,-1) -- (0,13);
        \draw[line width = 0.5mm] (0,10) -- (1,5) -- (4,4) -- (5,2) -- (10,1) -- (11,0);
    \end{tikzpicture}
    
    \caption{Một tập chuẩn đảo trong trường hợp hai chiều (được minh họa bằng phần màu hồng).}
    \label{fig:2}
\end{figure}

\begin{dn}[Tuy2000 \cite{Tuy2000}]
    Một tập $B \subset \R^p$ được gọi là một \textit{đa hộp (t.ư., đa hộp đảo)} trong hộp $[a, b]$ (t.ư., $[z, b]$), trong đó $z \in T$ và $T$ là một tập hữu hạn các đỉnh thuộc $[a, b]$.
\end{dn}

Khi đó ta gọi $T$ là tập đỉnh của đa hộp (t.ư., đa hộp đảo) $B$. Ta cũng
nói đa hộp (t.ư., đa hộp đảo) $B$ được sinh ra bởi $T$.

\begin{figure}[!h]
    \centering
    \begin{tikzpicture}[
        > = stealth, 
        line width = 0.3mm, 
        scale = 0.7
    ]
        \draw[fill = pink, fill opacity = 0.5, draw = none] (1,10) -- (1,5) -- (2,5) -- (2,3) -- (3,3) -- (3,1) -- (8,1) -- (8,0) -- (10, 0) -- (10, 10) -- cycle;
        \draw[->] (-1,0) -- (13,0);
        \draw[->] (0,-1) -- (0,13);
        \draw[line width = 0.5mm] (1,10) -- (1,5) -- (2,5) -- (2,3) -- (3,3) -- (3,1) -- (8,1) -- (8,0) -- (10, 0) -- (10, 10) -- cycle;
    \end{tikzpicture}
    \caption{ Một đa hộp đảo trong trường hợp hai chiều.}
    \label{fig:3}
\end{figure}

\begin{dn}[Tuy2000 \cite{Tuy2000}]
    Một đỉnh $z \in T$ được gọi là một đỉnh chính quy của đa hộp (t.ư., đa hộp đảo) $B$ nếu không tồn tại $z^\prime \in T, z^\prime \neq z$ sao cho $z^\prime$ (t.ư $z^\prime \leq z$). Hiển nhiên, một đỉnh $z \in T$ được gọi là \textit{không chính quy} khi nó không là đỉnh chính quy.
\end{dn}

\begin{md}[Tuy2000 \cite{Tuy2000}]
    Một đa hộp hoặc một đa hộp đảo được hoàn toàn xác định bởi các đỉnh chính quy của nó.
\end{md}

\begin{dl}[Tuy2000 \cite{Tuy2000}]
     Cho $f(x)$ là một hàm không giảm xác định trên đa hộp (t.ư., đa hộp đảo) $B$. Khi đó, $f(x)$ đạt giá trị cực đại (t.ư., cực tiểu) tại một đỉnh chính quy nào đó của $B$.
\end{dl}

\begin{md}[Tuy2000 \cite{Tuy2000}]
    Mọi đa hộp đều đóng và là tập chuẩn. Giao của một số hữu hạn các đa hộp là một đa hộp.
\end{md}
Mệnh đề sau đây chỉ ra cách xác định các đỉnh mới của một đa hộp đảo bằng kỹ thuật cắt đa hộp đảo được giới thiệu ở \cite{Tuy1999}
\begin{md}[Tuy 1999 \cite{Tuy1999}]
    Xét đa hộp đảo $[v, d]$ trong hộp $[b, d] \subset \R^p$ và một điểm $w$ thỏa mãn $v < w < d$. Khi đó tập $Q=[v, d] \setminus (w - int \R^p_+)$ là một đa hộp đảo có tập đỉnh được xác định như sau
    \begin{equation*}
        z^i = v + (w_i - v_i)e^i, \quad i = 1, \dots p,
    \end{equation*}
    trong đó $e$ là véc tơ đơn vị trong không gian $\R^p$
\end{md}

\begin{md} [Tuy 1999 \cite{Tuy1999}]
    \label{approx_normal_set}
     Có thể xấp xỉ một tập chuẩn (t.ư.,
tập chuẩn đảo) bằng một đa hộp (t.ư., đa hộp đảo) với một sai số nhỏ
bất kỳ.
\end{md}






\chapter{Bài toán quy hoạch giả lồi không trơn (NSP)}
Hai lớp hàm lồi tổng quá là tựa lồi và giả lồi xuất hiện thường xuyên trong các bài toán thực tế. Tuy nhiên, sự khuyết thiếu của tính lồi gây ra nhiều khó khăn trong việc giải các bài toán tối ưu giả lồi với các ràng buộc tựa lồi. Nhằm giải quyết vấn đề trên, Liu và các cộng sự \cite{Liu2021} đã đề xuất một phương pháp sử dụng hệ động lực giải xấp xỉ bài toán tối ưu giả lồi với ràng buộc tựa lồi. Trong đó, các chứng minh về sự hội tụ về nghiệm tối ưu được cung cấp một cách đầy đủ. \\
\indent Bài toán quy hoạch giả lồi không trơn (nonsmooth pseudoconvex programming) xét trong chương này được phát biểu như sau:
	\begin{align*}
	\tag*{NSP}
	\label{NSP}
    	\min\quad &r(x)   \\ 
    	s.t. \quad & x \in X,\nonumber 
	\end{align*}
trong đó, hàm mục tiêu $r$ là giả lồi không trơn, tập ràng buộc 
\begin{equation}
	\label{constraints_1}
	X = \{x \in \mathbb{R}^n\mid s(x)\leq 0\},\ s(x)=\left(s_1(x), s_2(x), \ldots, s_m(x)\right)^{\top}:\mathbb{R}^n \rightarrow \mathbb{R}^m
\end{equation}
 và $s_i(i=1, \ldots, m): \mathbb{R}^n \rightarrow \mathbb{R}$ các hàm tựa lồi khả vi.

\section{Cơ sở lý thuyết}

Xét bao hàm vi phân:
\begin{equation}
    \label{eq:inclusion}
    \dot x(t) \in F(x(t), t), \quad x(0) = x_0
\end{equation}
trong đó, $F(x(t), t)$ là ánh xạ đa trị. Ta gọi $x(.)$ là một trạng thái của \eqref{eq:inclusion} nếu $x(.)$ liên tục tuyệt đối với điều kiện ban đầu $x_0$ và tồn tại một hàm đo được $f(x(t), t) \in F(x(t), t)$ sao cho:
\begin{equation*}
    \dot x(t) = f(x(t), t), \text{ với hầu hết } t \in [0, T)
\end{equation*}



\begin{md}[\cite{Aubin1984}]
    \label{state_existance}
    Nếu ánh xạ đa trị $F$ trong \eqref{eq:inclusion} là nửa lên tục trên với tập giá trị là lồi compact, vậy thì tồn tại ít nhất một trạng thái $x(t): [0, T) \to \R^n$ của bao hàm vi phân \eqref{eq:inclusion}
\end{md}
\begin{bd}[\cite{Hosenini2016}] 
    \label{lemma 2.2}
    Giả sử $T (x)$ là một ánh xạ đa trị nửa liên tục trên,
vậy thì với mọii hàm liên tục $g(x)$, $\Theta(g(x))T (x)$ cũng là một ánh xạ nửa liên tục trên.
\end{bd}

\begin{md}[Chain rule \cite{Clarke1983}]
    \label{chain_rule}
    Cho $x(t): [0, \infty) \to \R^n, \ \mathcal{Y}: \R^n \to \R$. Nếu $x$ khả vi liên tục hầu khắp nơi và $\mathcal{Y}(x)$ là chính quy Lipschitz tại lân cận $x(t)$ thì,
    \begin{equation*}
        \dot{\mathcal{Y}}(x(t)) = \zeta^T \dot{x(t)}, \quad \forall \zeta \in \partial \mathcal{Y}(x(t)), \text{ với hầu hết $t \in [0, \infty)$}
    \end{equation*}
\end{md}

\begin{dn}[\cite{gen_convex}]
    \label{psuedomonotone_def}
    Cho tập lồi khác rỗng $\Lambda \subseteq \R^n$, ánh xạ đa trị $F: \Lambda \rightrightarrows \R^n$ được gọi là \textit{giả đơn điệu tăng} trên $\Lambda$ nếu với bất kỳ $x, \tilde{x} \in \Lambda$
    \begin{equation*}
        \exists \eta_x \in F(x): \eta_x^T (\tilde{x} - x) \geq 0 \Rightarrow \forall \eta_{\tilde{x}} \in F(\tilde{x}): \eta_{\tilde{x}}^T (\tilde{x} - x) \geq 0.
    \end{equation*} 
\end{dn}
Hơn nữa, một ánh xạ liên tục là giả lồi khi và chỉ khi dưới vi phân của nó là ánh xạ giả đơn điệu.
\begin{bd}[\cite{Liu2021}]
    \label{opimal_solution_lemma}
    Giả sử $x^*$ là nghiệm tối ưu của bài toán \eqref{NSP}. Với mọi $x \in X$,
    \begin{equation*}
        \xi^T (x - x^*) \geq 0, \quad \forall \xi \in \partial r
    \end{equation*}
\end{bd}
\begin{cm}
    Từ mệnh đề \ref{normal_cone_prop} kết hợp với giả thiết $x^*$ là nghiệm tối ưu của bài toán \eqref{NSP}, ta có $0 \in N_X(x^*) + \partial r(x^*)$. Tức là tồn tại $\xi^* \in \partial r(x^*)$ thỏa mãn
    \begin{equation} \label{hehe}
        -\xi^* \in N_X(x^*).
    \end{equation} 
    Vì $s_i$ là hàm tựa lồi, từ định lý \ref{quasiconvex_theo}, tập mức dưới của $s_i$ là tập lồi. Vì thế, miền chấp nhận được $X$ là tập lồi. Từ mệnh đề \ref{normal_cone_prop} và \eqref{hehe}, ta có $\xi^T (x - x^*) \geq 0,\ \forall x \in X$. Hơn nữa, từ định nghĩa \ref{psuedomonotone_def}, ta có $\partial r$ là hàm giả đơn điệu. Như vậy, với $x \in X$ bất kỳ, 
    $$ \xi^T(x - x^*)  \geq 0, \quad \forall \xi \in \partial r$$.
\end{cm}

\begin{dn}[\cite{Liu2021}]
    Một ánh xạ đa trị $\mathcal{V}: \Lambda \subseteq \R^n \rightrightarrows \R^n$ được gọi là nửa liên tục trên tại $x \in \Lambda$ nếu với bất kỳ tập mở $\mathcal{A} \supset \mathcal{V}(z)$, tồn tại lân cận $\mathcal{B}$ của $z$ thỏa mãn $\mathcal{V}(\mathcal{B}) \subseteq \mathcal{A}$. $\mathcal{V}$ nửa liên tục trên trên $\Lambda$ nếu $\mathcal{V}$ nửa liên tục trên tại mọi điểm thuộc $\Lambda$.
\end{dn}

\section{Phương pháp hệ động lực giải (NSP)}
Để giải bài toán \eqref{NSP}, ta định nghĩa ánh xạ đa trị $\psi: \R \rightrightarrows [0,1]$ như sau:
\begin{equation}
    \label{Psi}
    \Psi(s) = 
    \begin{cases}
        1, &s > 0\\
        [0, 1], & s= 0\\
        0, & s < 0
    \end{cases}
\end{equation}
Dễ thấy $\psi (s) = \partial \max \{0, s\}$, và $\Psi$ nửa lên tục trên trên $\R$. Ta tiếp tục định nghĩa hàm $P(x)$ như sau:
\begin{equation}
    P(x) = \sum_{i=1}^m \max \{0, s_i(x) \},
\end{equation}
trong đó, các hàm $s_i$ được cho trong \eqref{constraints_1}. Mặt khác, ta có bổ đề sau:
\begin{md}[\cite{Clarke1983}]
    Cho $G: \R^n \to \R$ là một hàm chính quy. Hàm $\max \{0, G(x) \}$ cũng là một hàm chính quy.
\end{md}
\indent Như vậy $P(x)$ là hàm chính quy. Hơn nữa, ta có thể biểu diễn $\partial P(x)$ như sau:
\begin{equation}
    \label{dP}
    \partial P(x) = \sum_{i=1}^m \Psi(g_(x)) \nabla g_i(x).
\end{equation}
Với các phân tích ở trên, Liu và các cộng sự \cite{Liu2021} đã đề xuất một mô hình hệ động lực được viết dưới dạng bao hàm vi phân hay hệ động lực để giải bài toán \eqref{NSP}:
\begin{equation}
    \label{model}
    \begin{cases}
        x(0) &\in X \\
        \dot x(t) &\in-c(x(t)) \partial r(x(t))-\partial S(x(t))
    \end{cases}
\end{equation}
trong đó,
\begin{equation}
    c(x(t))=\left\{\prod_{i=1}^{m} c_i(t) \mid c_i(t) \in 1-\Psi\left(s_i(x(t))\right), i=1,2, \ldots, m\right\}
\end{equation}
\indent Trong hệ động lực \eqref{model}, đại lượng $-\partial r(x(t))$ có tác dụng ép trạng thái $x(t)$ đi theo hướng giảm của hàm mục tiêu $r(x)$, đại lượng $\partial S(x(t))$ có tác dụng kéo trạng thái $x(t)$ về tập chấp nhận được $X$, $c(x(t))$ được sử dụng để điều chỉnh hướng đi của trạng thái $x(t)$. Như vậy, nếu tại một thời điểm nào đó $x(t) \notin X$, mô hình \eqref{model} sẽ kéo $x(t)$ trở lại tậm chấp nhận được $X$. Còn nếu $x(t) \in X$, mô hình sẽ đưa trạng thái đó đi theo hướng giảm chấp nhận được của $r(x(t))$.

\begin{bd} [\cite{Liu2021}]
    \label{lemma 2.2}
    Giả sử ánh xạ đa trị $\mathcal{U}$ là nửa liên tục trên với tập giá trị lồi compact, và hàm đa trị $\Psi$ được định nghĩa trong \eqref{Psi}. Với bất kỳ hàm liên tục $\mathcal{V}(x): \R^n \to \R$, $\Psi(\mathcal{V}(x))\mathcal{U}(x)$ là một ánh xạ đa trị nửa liên tục trên với miền giá trị lồi compact.
\end{bd}
\begin{cm}
    Dễ thấy với mọi $x \in \R^n$ thì tập $\Psi(\mathcal{V}(x))\mathcal{U}(x)$ là đóng và bị chặn, do đó đây là một tập compact. Từ bổ đề \ref{lemma 2.2}, với mọi ánh xạ liên tục $\mathcal{V}(x)$, $\Psi(\mathcal{V}(x))\mathcal{U}(x)$ là ánh xạ nửa liên tục trên. Chúng ta chỉ còn phải chỉ ra rằng $\Psi(\mathcal{V}(x))\mathcal{U}(x)$ là một ánh xạ đa trị với miền giá trị là tập lồi. \\
    \indent Thật vậy, với mỗi cặp $\mu_1, \mu_2 
    \in \Psi(\mathcal{V}(x))\mathcal{U}(x)$, tồn tại $\theta_1, \theta_2 \in \Psi(\mathcal{V}(x))$ và $\nu_1, \nu_2 \in \mathcal{U}(x)$ sao cho,
    $$ \mu_1 = \theta_1 \nu_1, \ \mu_2 = \theta_2 \nu_2 $$
    Ta có, 
    \begin{align*}
        \alpha \mu_1 + (1- \alpha) \mu_2 &= \alpha  \theta_1 \nu_1 + (1 - \alpha) \theta_2 \nu_2\\
        &= (\alpha \theta_1 + (1-\alpha) \theta_2) \left( \dfrac{\alpha  \theta_1 \nu_1}{\alpha \theta_1 + (1-\alpha) \theta_2} + \dfrac{(1 - \alpha) \theta_2 \nu_2}{\alpha \theta_1 + (1-\alpha) \theta_2} \right),
    \end{align*}
    với giả thiết $\mathcal{U}(x)$ lồi, ta có $\dfrac{\alpha  \theta_1 \nu_1}{\alpha \theta_1 + (1-\alpha) \theta_2} + \dfrac{(1 - \alpha) \theta_2 \nu_2}{\alpha \theta_1 + (1-\alpha) \theta_2} \in \mathcal{U}(x)$\\
    \indent Từ định nghĩa của $\Psi$, ta có $\Psi(\mathcal{V}(x))$ cũng là hàm lồi, cho nên $\alpha \theta_1 + (1-\alpha) \theta_2 \in \mathcal{V}(x)$. Do đó, 
    $$  \alpha \mu_1 + (1- \alpha) \mu_2 \in \Psi(\mathcal{V}(x))\mathcal{U}(x),\ \forall \alpha \in [0,1]$$.
\end{cm}

\begin{dl}[\cite{Liu2021}]
    \label{existance_theo}
    Với trạng thái ban đầu $x(0) \in X$ bất kỳ, tồn tại $T >0$ sao cho trạng thái $x(t): [0, T) \to \R^n$ của mô hình \eqref{model} tồn tại và nằm trong $X$. 
\end{dl}

\begin{cm}
    Từ bổ đề \ref{convex_compact}, ta có $\partial r$ là một ánh xạ đa trị nửa liên tục trên với miền giá trị lồi compact. Kết hợp với bổ đề \ref{lemma 2.2}, ta có vế phải của mô hình \eqref{model} là ánh xạ đa trị nửa liên tục trên với miền giá trị lồi compact. Do đó, theo mệnh đề \ref{state_existance}, tồn tại ít nhất một trạng thái $x(t): [0, T) \to \R^n$ của mô hình \eqref{model}.\\
    \indent Vì $x(t)$ là một trạng thái của \eqref{model}, tồn tại $\psi_i(t) \in \Psi(s_i(x(t))), \xi(t) \in \partial r(x(t)), \gamma(t) \in \partial P(x(t))$ thỏa mãn,
    \begin{equation}
        \dfrac{d}{dt} x(t) =-\left\{ \prod_{i=1}^{m} (1 - \psi_i(t)) \right\} \xi(t) - \gamma (t), \quad \text{với hầu hết $t \in [0, T)$}.
    \end{equation}
    Theo mệnh đề \ref{chain_rule}, ta có thể tính đạo hàm của $P(x(t))$ như sau:
    \begin{equation}
        \label{dx}
        \dfrac{d}{dt}P(x(t)) = \gamma(t)^T \left\{ -\left\{ \prod_{i=1}^{m} (1 - \psi_i(t)) \right\} \xi(t) - \gamma (t) \right\},\quad \text{với hầu hết $t \in [0, T)$}.
    \end{equation}
    \indent Giả sử tồn tại $t^\prime > t$ sao cho $x(t) \in X$ và $x(t^\prime) \notin X, \forall t^\prime \in (t, t + \Delta t]$. Từ định nghĩa của $P(x)$ ta có,
        \begin{align} 
            P(x(t)) &= 0 \\
            P(x(t^\prime)) &> 0,\quad \forall t^\prime \in (t, t + \Delta t]
        \end{align}
    Mặt khác, từ định nghĩa của $\Psi$ trong \eqref{Psi}, ta có
    \begin{equation}
        \prod_{i=1}^{m} (1 - \psi_i(t^\prime)) = 0, \quad \forall t^\prime \in (t, t + \Delta t]
    \end{equation}
    Từ đó kết hợp với (2.9) ta được,
    \begin{equation}
        \dfrac{d}{dt} P(x(t^\prime)) = \gamma(t)^T \left\{ - \gamma (t) \right\} = -\Vert \gamma(t^\prime) \Vert^2 \leq 0,\quad \text{với hầu hết $t \in (t, t + \Delta t]$}.
    \end{equation}
    Điều này kéo theo $P(x(t^\prime)) \leq P(x(t)) = 0$, (mâu thuẫn với 2.11). \\
    \indent Vì vậy, $x(t) \in X,\ \forall t \in [0, T).$
\end{cm}

\begin{dl}[\cite{Liu2021}]
    \label{bounded_theo}
    Cho $x(0) \in X$, trạng thái $x(t): [0, T) \to \R^n$ của mô hình \eqref{model} bị chặn, và do đó tồn tại ít nhất một trạng thái $x(t): [0, T) \to \R^n$ của mô hình \eqref{model} với trạng thái xuất phát $x(0)$ bất kỳ.
\end{dl}
\begin{cm}
    Cho $x^*$ là nghiệm tối ưu của bài toán \eqref{NSP}, theo định lý \ref{existance_theo}, ta có $x(t) \in X $ với mọi $t \in [0, T)$. Cho nên,
    \begin{align}
        \label{dnorm}
        \dfrac{1}{2}\dfrac{d}{dt} \Vert x(t) - x^* \Vert^2 &= - (x(t) - x^*)^T \left( \prod_{i=1}^{m}(1 - \psi_i(t)) \right) \xi(t)\\
     &- (x(t) - x^*)^T \gamma(t) \nonumber
    \end{align}
    với hầu hết $t \in [0, T)$, trong đo $\psi_i(t), \xi(t), \gamma(t)$ được định nghĩa như trong \eqref{dx}. \\
    \indent Mặt khác, từ bổ đề \ref{opimal_solution_lemma} và $\prod_{i=1}^{m}(1 - \psi_i(t)) \geq 0$, kéo theo
    \begin{equation}
        - (x(t) - x^*)^T \left( \prod_{i=1}^{m}(1 - \psi_i(t)) \right) \xi(t) \leq 0
    \end{equation}
    Ký hiệu $I_0(x):= \{i \in \{1, \dots, m \}: s_i(x) = 0 \}$. Như vậy, do $x^*$ thuộc tập chấp nhận được ta hiển nhiên có:
    $$ s_i(x^*) \leq s_i(x(t)),\ \forall i \in I_0(x(t)) $$.
    Như vậy, từ định nghĩa \ref{psuedoconvex_def} ta có
    $$ \nabla s_i(x(t))^T (x(t) - x^*) \geq 0,\ \forall i \in I_0(x(t))$$
    Tiếp theo, từ \eqref{dP}, ta có tồn tại các bộ số $\lambda_i(t) \in [0,1],\ i \in I_0(x(t)) $ sao cho
    \begin{equation}
        (x^* - x(t))^T\gamma(t) = (x^* - x(t))^T\left( \sum_{i \in I_0(x(t))} \lambda_i(t)\nabla s_i(x(t)) \right) \leq 0
    \end{equation}
    Do đó, 
    \begin{equation}
        \label{lessthan0}
        \dfrac{1}{2}\dfrac{d}{dt} \Vert x(t) - x^* \Vert^2 \leq 0
    \end{equation}
    kéo theo 
    \begin{equation}
        % \label{x_limit}
        \Vert x(t) - x^* \Vert^2 \leq \Vert x(0) - x^* \Vert^2.\ \forall t \in [0, T).
    \end{equation}
    Cho nên, các trạng thái $x(t): [0, T) \to \R^n$ của mô hình \eqref{model} với $x(0) \in X$ bị chặn.
\end{cm}  




\section{Sự hội tụ của thuật toán}
\begin{bd}[\cite{Liu2021}]
    \label{bd 2.3.1}
    Cho mô hình \eqref{model} với trạng thái $x(0) \in X$, tồn tại số thực $\alpha > 0 $ sao cho với bất kỳ $c_i(t) \in \Psi(s_i(x(t)))$,
    \begin{equation}
        \label{hihi}
        \prod_{i=1}^m c_i(t) \geq \alpha > 0
    \end{equation}
\end{bd}
\begin{cm}
    Giả sử \eqref{hihi} không đúng, tồn tại dãy $\{t_n\} \subseteq (0, \infty)$, $\psi_i(s_i(x(t_n)))$ sao cho
    \begin{equation}
        \label{hihi1}
        0 \leq \prod_{i=1}^m (1 - \psi_i(t_n)) \leq \dfrac{1}{n}.
    \end{equation}
    Do $\{x(t_n) \}$, $\{\psi_i(t_n) \}$ đều là hai dãy bị chặn và $X$ là tập đóng, cho nên tồn tại hai dãy con $\{ x(t_{n_s})\}$, $\{\psi(t_{n_s})\}$ cùng với $\tilde{x} \in X$ và $\tilde{\psi_i \in [0,1]},\ (i=1, \dots, m)$ thỏa mãn
    \begin{equation}
        \begin{aligned}
            \lim_{n_s \to \infty} x(t_{n_s}) &= \tilde{x} \\
            \lim_{n_s \to \infty} \psi_i(t_{n_s}) &= \tilde{\psi},\ i = 1, \dots, m
        \end{aligned}
    \end{equation}
    Để ý rằng $\Psi(s_i(.))$ là hàm nửa liên tục trên và $\psi_i(t_{n_s}) \in \Psi(s_i(x(t_{n_s})))$ cho nên $\tilde{\psi} \in \Psi(s_i(\tilde{x})),\ i = 1, \dots, m$.
    Tại \eqref{hihi1}, cho $n_s \to \infty$, ta có 
    \begin{equation}
        \lim_{n_s \to \infty}(1 -  \psi_i(t_{n_s})) = \prod_{i=1}^m (1 - \tilde{\psi}_i) = 0,
    \end{equation}
    hay tồn tại ít nhất một $i_0 \in \{1, \dots, m \}$ sao cho $\tilde{\psi}_{i_0} = 1$ nghĩa là $s_{i_0} > 0$. Điều này mâu thuẫn với điều kiện ràng buộc của bài toán \eqref{NSP}. Suy ra, \eqref{hihi} đúng.
\end{cm}

\begin{dl}[\cite{Liu2021}]
    Trạng thái $x(t) \in X$ của mô hình \eqref{model} hội tụ đến nghiệm tối ưu của bài toán \eqref{NSP}.
\end{dl}

\begin{cm}
    Từ định lý \ref{existance_theo}, ta có $x(t) \in X,\ \forall t \leq 0$. Đặt $x^*$ là một nghiệm tối ưu của bài toán tối ưu \eqref{NSP}, và xét hàm $V(x)$ như sau:
    \begin{equation}
        V(x) = \dfrac{1}{2} \Vert x(t) - x^* \Vert^2.
    \end{equation}
    Trước hết, ta chỉ ra rằng:
    \begin{equation}
        \label{dv}
        \limsup_{t \to \infty} \dfrac{dV(x(t))}{dt} = 0.
    \end{equation}
    Giả sử sai, theo \eqref{lessthan0}, tồn tại số $\kappa > 0$ thỏa mãn
    \begin{equation}
        \dfrac{dV(x(t))}{dt} \leq -\kappa \text{ với hầu hết } t > 0.
    \end{equation}
    Tích phân hai vế từ $0$ đến $t$ ta nhận được
    \begin{equation}
        V(x(t)) - V(x(0)) = \int_{0}^{t} \dfrac{dV(x(s))}{ds} ds \leq -\kappa t \xrightarrow{t \to \infty} -\infty 
    \end{equation}
    Điều này mâu thuẫn vì $V(x) \leq 0$. Cho nên, \eqref{dv} đúng và ta có dãy con $\{t_n \} \subseteq [0, \infty)$ thỏa mãn
    \begin{equation}
        \label{lim_d}
        \lim_{n \to \infty} \dfrac{dV(x(t_n))}{dt} = 0.
    \end{equation}
    Theo định lý \ref{bounded_theo}, $\{x(t_n)\}$ bị chặn và do đó tồn tại một dãy con $\{x(t_{n_s}) \}$ và $\breve{x}\in X $ sao cho
    \begin{equation}
        \label{x_limit}
        \lim_{n_s \to \infty} x(t_{n_s}) = \breve{x}.
    \end{equation}
    Thêm vào đó, vì các ánh xạ đa trị $\Psi, \partial r$ và $\partial P$ là nửa liên tục trên, từ \eqref{x_limit}, tồn tại các $\psi_i(t_n) \in \Psi(s_i(x(t_n))), \xi(t_n) \in \partial r(x(t_n)), \gamma(t_n) \in \partial P(x(t_n))$ thỏa mãn
    \begin{equation}
        \label{limits}
        \begin{aligned}
            \lim_{n \to \infty} \psi_i(t_n)  &= \breve{\psi_i} \in \Psi(s_i(\breve{x})),\ i = 1, \dots, m,\\
            \lim_{n \to \infty} \xi(t_n) &= \breve{\xi} \in \partial r(\breve{x}), \\
            \lim_{n \to \infty} \gamma_(t_n) &= \breve{\gamma} \in \partial P(\breve{x}).
        \end{aligned}
    \end{equation}
    Theo \eqref{dnorm}, \eqref{lim_d} và \eqref{limits}, ta có
    \begin{equation}
        \label{hoho}
        \begin{aligned}
            0 &= \lim_{n \to \infty}(x(t_n) - x^*)^T 
            \left\{ \left(\prod_{i=1}^m (1 - \psi_i(t_n)) \right) \xi(t_n) + \gamma(t_n) \right\} \\
            &= (\breve{x} - x^*)^T \left\{\left( \prod_{i=1}^m (1 - \breve{\psi}_i)  \right) \breve{\xi} + \breve{\gamma}  \right\}
        \end{aligned}
    \end{equation}
    Mặt khác, bổ đề \ref{bd 2.3.1} cho thấy
    \begin{equation}
        \prod_{i=1}^m (1 - \psi_i(t_n)) \geq \alpha > 0, \quad \forall  \in \mathbb{N}.
    \end{equation}
    Cho $n \rightarrow \infty$ ta có
    \begin{equation*}
        \prod_{i=1}^m (1 - \breve{\psi}) \geq \alpha > 0.
    \end{equation*}
    Tiếp theo, ta sẽ chỉ ra rằng $\breve{x}$ là một nghiệm tối ưu của bài toán \eqref{NSP}. Giả sử sai, tồn tại $x^* \in X$ thỏa mãn $r(x^*) < r(\breve{x})$. Do $r$ là hảm giả lồi, nên
    $$ \breve{\xi}^T (\breve{x} - x^*) \geq 0, \quad \forall \breve{\xi} \in \partial r(\breve{x})$$.
    Kết hợp điều trên với \eqref{hoho}, ta nhận được
    \begin{equation}
        \label{lessthan0_1}
        (\breve{x} - x^*)^T\breve{\gamma} < 0,
    \end{equation}
    Tuy nhiên, để ý rằng $s_i(x^*) \leq s_i(\breve{x}) = 0,\ \forall i \in i_0(\breve{x})$, từ tính tựa lồi của các hàm $s_i$ và mệnh đề \ref{md_quasi} ta có
    $$ \nabla s_i(\breve{x})(\breve{x} - x^*) \geq 0,\ \forall i \in I_0(\breve{x}).$$
    Tiếp theo, theo \eqref{dP}, tồn tại bộ số $\breve{\lambda}_i \in [0,1]$, $i \in I_0(\breve{x})$ sao cho
    $$ (\breve{x} - x)^T \breve{\gamma} \left(\sum_{i \in I_0(\breve{x})} \breve{\lambda}_i\nabla s_i(\breve{x}) \right) \geq 0.$$
    Điều này mâu thuẫn với \eqref{lessthan0_1}. Do đó, $\breve{x}$ là một nghiệm tối ưu của bài toán \eqref{NSP}.\\
    \indent Cuối cùng, ta chứng minh $\lim_{n \to \infty} x(t) = \breve{x}$. Ta định nghĩa hàm 
    $$\breve{V(x)} = \dfrac{1}{2} \Vert x - \breve{x} \Vert^2.$$  
    Ta có thể dễ dàng chỉ ra rằng 
    \begin{equation*}
        \dfrac{d}{dt} \breve{V(x(t))} \leq 0,\ \text{với hầu hết } t \geq 0,
    \end{equation*}
    kéo theo $\lim_{n \to \infty} \Vert x(t) - \breve{x}$ tồn tại. Từ \eqref{x_limit} ta được
    \begin{equation*}
        \lim_{t \to \infty} \Vert x(t) - \breve{x} \Vert^2 = \lim_{n \to \infty} \Vert x(t_n) - \breve{x} \Vert^2 = 0.
    \end{equation*}
\end{cm}




\section{Thử nghiệm tính toán}
Xét ví dụ được trình bày trong nghiên cứu của Liu và các cộng sự \cite{Liu2021},
\begin{equation}
    \label{vdchap2}
    \begin{aligned}
        \min\ & f(x) = \dfrac{x_1^2 + x_2^2 +3}{1 + 2x_1 + 8x_2}\\
         s.t\ & g(x) = -x_1^2 - 2x_1x_2 \leq -4,\\
              & x_1, x_2 \geq 0.
    \end{aligned}
\end{equation}
Hình 2.1 mô tả quá trình thay đổi trạng thái của hệ động lực \eqref{model} với các điểm xuất phát khác nhau. Ta thấy rằng thuật toán hội tụ về nghiệm tối ưu $x^* = (0.8907, 1.8009)^T$.
\begin{figure}[H]
    \centering
    \includegraphics*[width=\textwidth]{Pictures/ex_chapter2.png}
    \caption{Quá trình thay đổi trạng thái của mô hình \eqref{model} với các trạng thái ban đầu khác nhau hội tụ về nghiệm của bài toán \eqref{vdchap2}}
\end{figure}
\chapter{Bài toán tối ưu đa mục tiêu hai cấp tựa lồi (QBP)}
Chương này của đồ án sẽ đưa mối liên hệ giữa bài toán tối ưu hai cấp tựa lồi (QBP) được xét và bài toán quy hoạch giả lồi không trơn \eqref{NSP}. Qua đó, xây dựng một lược đồ nhánh cận với cốt lõi là phương pháp hệ động lực và thuật toán xấp xỉ trong dựa trên đa hộp đảo nhằm đưa ra nghiệm tối ưu.

\section{Phát biểu bài toán}
Bài toán tối ưu hai cấp tựa lồi được xét trong đồ án này được phát biểu như sau:
\begin{align}
    \min \  &{h}(x,y) \tag*{(QBP)}\label{prob_QBP} \\
    \text{s.t.} \quad & {g}(x,y) \leq 0, y\in\mathbb{R}_{+}^{m}, \nonumber\\ 
                     & x \in {\text{Argmin}}\{f(x) \mid x\in \mathcal{X}\},  \nonumber
\end{align}
trong đó, các dữ kiện đi kèm bao gồm:\\
\noindent \textbf{(A1)} $\mathcal{X} = \{x \in \mathbb{R}^n \mid s(x)\leq 0 \}$ là tập lồi khác rỗng và bị chặn; 

\noindent \textbf{(A2)} Hàm vector ${h}: \mathbb{R}^m \times \mathbb{R}^n \rightarrow \mathbb{R}$ liên tục và các hàm vector ${g}: \mathbb{R}^m \times \mathbb{R}^n \rightarrow \mathbb{R}^{\ell}, {f}: \mathbb{R}^n \rightarrow \mathbb{R}^p, {s}: \mathbb{R}^n \rightarrow \mathbb{R}^q$ khả vi liên tục, $m, n, \ell, p , q \geq 2$;

\noindent \textbf{(A3)} Các hàm mục tiêu $h, f$là giải lồi và chính quy; 

\noindent \textbf{(A4)} Các hàm ràng buộc $g, s$ là tựa lồi;

Bài toán \ref{prob_QBP}
bao hàm một lớp lớn các bài toán tối ưu lồi bao gồm bài toán quy hoạch tích lồi, quy hoạch song tuyến tính và quy hoạch toàn phương. \\
\indent Định nghĩa tập ảnh $\mathcal{Z}:=\{z\in\mathbb{R}^{p}\mid z=f(x),x\in X\}$ và $\mathcal{Z}^{+}:=\mathcal{Z}+\mathbb{R}_{+}^{p}=\{z\in\mathbb{R}^{p}\mid\exists z^{0}\in\mathcal{Z},z^{0}\leq z\}$ và tập $\mathcal{G}=\left\{(x,y)\mid x\in X,y\in\mathbb{R}_{+}^{m},g(x,y)\le0\right\}$.\\
\indent Vì $\mathcal{X}$ bị chặn và $f$ bao gồm các hàm liên tục, $\mathcal{Z}$ bị chặn. Ta vì thế có thể xác định một hộp $[m,M] \subset \R^p$ chứa $\mathcal{Z}$, i.e., $m\leq\mathcal{Z}\leq M$. Để xác định $m$, ta có thể giải bài toán sau mới mỗi $m_{i},i=1,2,\dots,p$:
\begin{equation}\label{P_i}
    \min\;f_{i}(x),\;\;{\rm s.t.}\;x\in \mathcal{X}.    \tag*{\ensuremath{(P^i)}}
\end{equation}
Vì $f_{i}(x)$ là hàm giả lồi, bất kỳ cực tiểu địa phương đều là cực tiểu toàn cục. Ta hoàn toàn có thể áp dụng thuật toán hệ động lực được trình bày trong chương 2 để giải \ref{P_i}. Tuy nhiên, lập luận tương tự không thể được áp dụng để tìm $M$ bởi vì các bài toán
$\max\;f_{i}(x),\;\;{\rm s.t.}\;x\in \mathcal{X}$ là không lồi. Thay vào đó, chúng ta bao $\mathcal{X}$ trong một tập đơn hình $\Delta$ tập đỉnh
$V(\Delta)=\{\Delta^{0},\Delta^{1},\dots,\Delta^{n}\}$, trong đó $\Delta^{0}=\min\{x\in\mathbb{R}^{n}\mid x\in \mathcal{X}\}$
và $\Delta^{i}=(\Delta_{1}^{i},\Delta_{2}^{i},\dots,\Delta_{n}^{i}),i=1,2,\dots,n$ được đỉnh nghĩa bởi
\[
\Delta_{k}^{i}=\begin{cases}
\Delta_{k}^{0}, & {\rm if}\;k\neq i\\
U-\sum_{j\neq k}\Delta_{j}^{0}, & {\rm if}\;k=i,
\end{cases}
\]
trong đó $U=\max\{\langle e,x\rangle\mid x\in X\},e\in\mathbb{R}^{n},e=(1,1,\dots,1)^{T}$. Điều này dẫn đến $X\subset\Delta$ và vì thế ta có một cách để xác định $M$
\begin{align*}
M_{i}: & =\max\{f_{i}(x)\mid x\in V(\Delta)\}\\
 & =\max\{f_{i}(x)\mid x\in\Delta\}\\
 & \geq\max\{f_{i}(x)\mid x\in X\},i=1,2,\dots,p.
\end{align*}

Bao tập $\mathcal{Z}^{+}$ bởi một tập hữu hiệu tương đương với $\mathcal{Z}$
\begin{align*}
\mathcal{Z}^{\diamond}:=\; & \mathcal{Z}^{+}\cap(M-\mathbb{R}_{+}^{p}).
\end{align*}

\begin{md} \label{lem_MinY} Ta có
    \begin{itemize}
        \item[i] $\;{\rm Min}\mathcal{Z}={\rm Min}\mathcal{Z}^{+}={\rm Min}\mathcal{Z}^{\diamond}$; 
        \item[ii] ${\rm WMin}\mathcal{Z}={\rm WMin}\mathcal{Z}^{+}\cap\mathcal{Z}={\rm WMin}\mathcal{Z}^{\diamond}\cap\mathcal{Z}$.
    \end{itemize}
\end{md}

Ta sẽ chuyển đổi bài toán \ref{prob_QBP} về một bài toán tối ưu với hàm mục tiêu đơn điệu giảm bằng việc định nghĩa hàm $\varphi:[m,M] \longrightarrow\mathbb{R}$ như sau:
\[
\varphi(z)=\min\{h(x,y)\mid (x,y)\in \mathcal{G},f(x)\leq z\}.
\]
Như ta có thể thấy, khi $y$ tăng, $\varphi(z)$ giảm do sự dãn ra của không hữu hiệu $\mathcal{X}$, vì vậy $\varphi$ là một hàm giảm.
\begin{md}\label{prop-QWP_Y}Bài toán \ref{prob_QBP} tương đương với bài toán tối ưu trên không gian ảnh
    \begin{equation}
    \min\{\varphi(z)\mid z\in{\rm WMin}\mathcal{Z}^{+}\cap[m,M]\}\tag*{(OP)}\label{prob_OP}
    \end{equation}
\end{md}
\begin{cm}
    Bài toán \ref{prob_QBP} tương đương với
    \[
    \begin{array}{cl}
     & \min\{h(x,y)\mid g(x,y)\le0,x\in X_{WE},y\in\mathbb{R}_{+}^{m}\}\\
    \iff & \min\{h(x,y)\mid x\in X,y\in\mathbb{R}_{+}^{m},g(x,y)\le0,f(x)\in{\rm WMin}\mathcal{Z}\}\\
    \iff & \min\{h(x,y)\mid (x,y)\in\mathcal{G},f(x)\le z,z\in{\rm WMin}\mathcal{Z},z\in[m,M]\}\\
    \iff & \min\{\varphi(z)\mid z\in{\rm WMin}\mathcal{Z}\cap[m,M]\}.
    \end{array}
    \]
    
    Bây giờ ta sẽ chứng minh nếu $\bar{z}$ là nghiệm tối ưu của bài toán $\min\{\varphi(z)\mid z\in{\rm WMin}\mathcal{Z}^{+}\cap[m,M]\}$
    và $(\bar{x},\bar{y})$ là nghiệm tối ưu của bài toán $\varphi(\bar{z})$, thì $f(\bar{x})$ là nghiệm tối ưu của $\min\{\varphi(z)\mid z\in{\rm WMin}\mathcal{Z}\cap[m,M]\}$ và vì thế \ref{prob_QBP} có nghiệm tối ưu duy nhất $(\bar{x},\bar{y})$.\\
    \indent Vì $(\bar{x},\bar{y})$ là nghiệm tối ưu của $\varphi(\bar{z})$
    \begin{equation}
    \begin{cases}
    \varphi(\bar{z})=h(\bar{x},\bar{y})\\
    f(\bar{x})\le\bar{z}\\
    h(\bar{x},\bar{y})\le h(x,y) & \forall (x,y)\in\mathcal{G},f(x)\le\bar{z},
    \end{cases}\label{eq:hx_1}
    \end{equation}
    vì thế ta có
    \[
    \left\{(x,y)\in\mathcal{G}\mid f(x)\le f(\bar{x})\right\} \subseteq\left\{(x,y)\in\mathcal{G} \mid f(x)\le \bar{z}\right\},
    \]
    
    kết hợp với bất đẳnt thức thứ 3 của \eqref{eq:hx_1} ta có
    
    \[
    h(\bar{x},\bar{y})\le h(x,y)\;\;\;\;\;\forall (x,y)\in\mathcal{G},f(x)\le f(\bar{x}).
    \] 
    Vì thế $(\bar{x},\bar{y})$ cũng là nghiệm tối ưu của $\varphi(f(\bar{x}))$ và
    \begin{equation}
    \varphi(f(\bar{x}))=h(\bar{x},\bar{y}).\label{eq:hx_2}
    \end{equation}
    
    Từ bất đẳng thức thứ nhất của \eqref{eq:hx_1} và \eqref{eq:hx_2}, ta có
    \begin{equation}
    \varphi(f(\bar{x}))=\varphi(\bar{z}).\label{eq:fx_z}
    \end{equation}
    Vì $f(\bar{x})\le\bar{z}$ và $\bar{z}\in{\rm WMin}\mathcal{Z}^{+}\cap[m,M]$, ta có $f(\bar{x})\in{\rm WMin}\mathcal{Z}^{+}\cap[m,M]$. Mặt khác, $f(\bar{x})\in\mathcal{Z}$ vì vậy
    \begin{equation}
    f(\bar{x})\in{\rm WMin}\mathcal{Z}\cap[m,M].\label{eq:fx_WMinZ}
    \end{equation}
    Hơn nữa,
    \begin{equation}
    \left\{ z\in\mathcal{Z}\mid z\in{\rm WMin}\mathcal{Z}\cap[m,M]\right\} \subseteq\left\{ z\in\mathcal{Z}\mid z\in{\rm WMin}\mathcal{Z}^{+}\cap[m,M]\right\} ,\label{eq:WMinZ}
    \end{equation}
    
    Từ \eqref{eq:fx_z}, \eqref{eq:fx_WMinZ}, \eqref{eq:WMinZ} và
    do $\bar{z}$ là nghiệm tối ưu của $\min\{\varphi(z)\mid z\in{\rm WMin}\mathcal{Z}^{+}\cap[m,M]\}$, ta phải có $f(\bar{x})$ là nghiệm tối ưu của $\min\{\varphi(z)\mid z\in{\rm WMin}\mathcal{Z}\cap[m,M]\}$.\\
    \indent Kết luận, ta có thể giải \ref{prob_OP} nhằm mục đích giải \ref{prob_QBP}.
\end{cm}
\section{Thuật toán xấp xỉ trong}
Một tập $Q \in [m, M ]$ được gọi là tập chuẩn nếu
$[m, z] \in Q$, $\forall z \in Q$ và đa chuẩn nếu $[z, M ]  Q\in, \forall z \in Q$. Theo mệnh đề \ref{union_iter_normalset}, hợp của một họ hữu hạn các tập chuẩn (đa tập chuẩn) cũng là tập chuẩn (đa tập chuẩn). Hợp của mọi tập chuẩn (đa tập chuẩn) chứa trong $Q$ được gọi là \textit{bao chuẩn}
(\textit{bao chuẩn đảo}) của $Q$. Để cho thuận tiện, ta ký hiệu bao chuẩn (bao chuẩn đảo) của tập $Q$ bởi $\mathcal{N}\left(Q\right)$ ($\mathcal{M}\left(Q\right)$) đồng thời cũng là đa tập chuẩn nhỏ nhất chứa $Q$. Đa hộp $\mathcal{P}$ là bao chuẩn của hữu hạn các đỉnh $V\subset\left[m,M\right]$, i.e. $\mathcal{P}=\bigcup_{v\in V}\left[m,v\right]$
hoặc $\mathcal{P}=\mathcal{N}\left(V\right).$ \textit{đa hộp đảo} $\mathcal{P}$ là bao đa chuẩn của tập hữu hạn các đỉnh $V\subset\left[m,M\right]$, i.e. $\mathcal{P}=\bigcup_{v\in V}\left[v,M\right]$
hoặc $\mathcal{P}=\mathcal{M}\left(V\right).$\\
\indent Chương này của đồ án sẽ trình bày một thuật toán xấp xỉ trong cho tập $\mathcal{Z}^+$, ở đó, thay vì xác định một đa hộp đảo phía ngoài bởi các đỉnh chính quy $V$, ta sử dụng các đỉnh chính quy của đa hộp đảo tương ứng và xây dựng đa hộp đảo phía trong như sau: 
\[
\mathcal{L}\left(V\right)\coloneqq\left[m,M\right]\setminus{\rm int}\left(\mathcal{N}\left(V\right)-\mathbb{R}_{+}^{p}\right).
\]
Để cho thuận tiện, xuyên suốt phần còn lại của đồ án, đa hộp đảo sẽ được xây dựng theo cách phía trên. Đa hộp đảo phía trong $\mathcal{L}\left(V\right)$
được xây dưng như trên sẽ sinh ra các đỉnh chính quy mới và các đỉnh chính quy ban đầu $V$ sẽ được gọi là các đỉnh \textit{chính quy đảo}. Theo mệnh đề \ref{approx_normal_set}, ta có thể xấp xỉ một tập chuẩn đảo compact bởi một đa hộp đảo với sai số nhỏ bất kỳ. Vì vậy, tập chuẩn đảo compact của tập $\mathcal{Z}^{\diamond}$ có thể được xấp xỉ
bởi một họ các đa hộp đảo. Thuật toán dưới đây sinh một dãy các đa hộp đảo lồng nhau để xấp xỉ trong tập ảnh $\mathcal{Z}^{\diamond}$,
i.e.
\[
\mathcal{P}^{0}\subset\mathcal{P}^{1}\subset\mathcal{P}^{2}\subset\dots\subset\mathcal{P}^{k}\subset\mathcal{P}^{k+1}\subset\dots\subset\mathcal{Z}^{\diamond},
\]
trong đó đa hộp đảo xấp xỉ đầu $\mathcal{P}^{0}=\left[M,M\right]$ với $M$ được xác định như đã trình bày trong phần \ref{3.1}.\\
\indent Tại mỗi bước lặp, quá trình đa hộp đảo $\mathcal{P}^{k+1}$
được sinh ra từ $\mathcal{P}^{k}$ và đưa ra tập xấp xỉ trong 
được mô tả như trong thủ tục dưới đây\\

\begin{algorithm}[H]
\SetAlgorithmName{Procedure}{procedure}{List of Procedures}
\renewcommand{\thealgocf}{}
\let\oldnl\nl
\newcommand{\nonl}{\renewcommand{\nl}{\let\nl\oldnl}}
\caption{\textit{CopolyblockCut}}
\label{algo:next_time_step}
\KwIn{Một đa hộp đảo $V^k$ hoặc tập các đỉnh chính quy, đỉnh chính quy đang xét $v^k$ và giao điểm $w^k$ của tia xuất phát từ $v^k$ theo hướng $\hat{d}$ và $\partial\mathcal{Z^+}$}
\KwOut{Xấp xỉ trong của tập ảnh}
Đặt $V^{k+1}\gets V^{k}\setminus\{v^{k}\}$.\\
\For{$i \gets 1$ \textbf{\textup{to}} $p$}
{$z^{i} = v^{k}-(v^{k}_{i}-w^{k}_{i})e^{i}$.\\
\If{$z^{i}_{i}\neq m_{i}$}
{$V^{k+1}\gets(V^{k+1}\cup\{z^{i}\})$.}}
%\ForEach{$w\in V^k\setminus \{ v^k \}$}{
%	\If{$w\leq v^k$ and $\exists i \in \{1,\dots,p\}$ such that $w_i>w^k_i$}{
%		%\tcc{$w^k_i$ is an improper element}
%		Remove $w^k_i$.
%	}
%}
\end{algorithm}


\section{Phương pháp hệ động lực giải bài toán (QBP)}
Phần này của chương 3 sẽ áp dụng phương pháp hệ động lực để tìm cận trên và cận dưới cho lược đồ nhánh cận để giải bài toán \eqref{prob_QBP}.
\subsection{Xác định cận trên $\alpha$}
Một điểm hữu hiệu yếu của $\mathcal{Z}$ có thể dễ dàng được xác định bởi chú ý sau:\\
\begin{cy}\label{remark:P_0_v}Cho $\hat{d}>0$ là một vector trong
$\mathbb{R}^{n}$ và $v$ là một điểm bất kỳ $\mathbb{R}^{p}$.
Giao điểm $w_{v}$ của tia xuất phát từ $v$ theo hướng
$\hat{d}$ có thể được xác định bởi
\begin{equation} \label{eq:bar_w}
    w_{v}=v+t_{v}\hat{d}
\end{equation} 
trong đó $t_{v}$
là một nghiệm tối ưu của bài toán
\begin{equation}
\begin{array}{rl}
\min & t\tag*{\ensuremath{(P^{0}(v))}}\\
{\rm s.t.} & v+t\hat{d}\in\mathcal{Z}^{+},\;t\in\mathbb{R}.
\end{array}\label{eq:P_0_v}
\end{equation}
\end{cy}

\begin{bd} \label{lem_bar_w} 
    Cho $v$ là một điểm bất kỳ trong $\mathbb{R}^{p}$.
    Tồn tại duy nhất một điểm $w_{v}$ xác định bởi \eqref{eq:bar_w}
    và là điểm hữu hiệu yếu của $\mathcal{Z}^{+}$.  
\end{bd}

Ta có thể viết lại \ref{eq:P_0_v} dưới dạng tường minh
\begin{align}
\min\,\, & t\tag*{\ensuremath{(P^{1}(v))}}\label{eq:P_1_v}\\
\mbox{s.t.\,\,} & f(x)-t\hat{d}-v\leq0,\nonumber \\
 & x\in X,\;t\in\mathbb{R}.\nonumber 
\end{align}
Dễ thấy \ref{eq:P_1_v} là một bài toán không lồi. Tuy nhiên, nó lại tương đương với bài toán sau
\begin{align}
\min\,\, & \max\{\dfrac{f_{j}(x)-v_{j}}{\hat{d}_{j}}\mid j=1,...,p\}\tag*{\ensuremath{(P^{2}(v))}}\label{eq:Pbarv}\\
\mbox{s.t.}\;\; & x\in X.\nonumber 
\end{align}
và bài toán lúc này có thẻ giải bởi hàm mục tiêu là giả lồi và tập chấp nhận được là tập lồi.
\begin{bd}\cite{Thang2020}\label{lem_equiv} 
    Bài toán \ref{eq:P_0_v} và \ref{eq:Pbarv}
    tương đương, i.e. nếu $(x^{*},t^{*})$ là nhiệm tối ưu của
    \ref{eq:P_0_v} vậy thì $x^{*}$là nghiệm tối ưu của \ref{eq:Pbarv}.
    Ngược lại, nếu $x^{*}$ là nghiệm tối ưu và $t^{*}$ là nghiệm tối ưu của \ref{eq:Pbarv} thì $(x^{*},t^{*})$ là nghiệm tối ưu của \ref{eq:P_0_v}. Hơn nữa, bài toán \ref{eq:Pbarv} là quy hoạch giả lồi.
\end{bd}

Theo bổ đề \ref{lem_equiv}, việc tìm nghiệm cho bài toán \ref{eq:P_0_v} được chuyển thành việc tìm nghiệm tối ưu cho bài toán quy hoạch giả lồi.  Vì hàm mục tiêu của \ref{eq:Pbarv} là hàm không trơn cho dù các hàm ${f_i}$ trơn, phương pháp hệ động lực được sử dụng để giải quyết bài toán này. 

% \begin{proof}We consider this proof under the known equivalence of
% \ref{eq:P_0_v} and \ref{eq:P_1_v}. Let $(x^{*},t^{*})$ be the optimal
% solution of \ref{eq:P_1_v}. The feasible condition of \ref{eq:P_1_v}
% can be rewritten as
% \[
% t\geq\max\{\dfrac{f_{j}(x)-v_{j}}{\hat{d}_{j}}\mid j=1,\dots,p\},\;x\in\mathcal{S},\;t\in\Bbb R.
% \]

% Assuming there exists $x\in\mathcal{S}$ such that $$\max\{\dfrac{f_{j}(x)-v_{j}}{\hat{d}_{j}}\mid j=1,\dots,p\}<\max\{\dfrac{f_{j}(x^{*})-v_{j}}{\hat{d}_{j}}\mid j=1,\dots,p\}.$$Let $t=\max\{\dfrac{f_{j}(x)-v_{j}}{\hat{d}_{j}}\mid j=1,\dots,p\}$
% then $(x,t)$ is feasible and corresponds to a better objective value
% of \ref{eq:P_1_v}, which is contrary. Therefore, $$x^{*}=\min\{\max\{\dfrac{f_{j}(x)-v_{j}}{\hat{d}_{j}}\mid j=1,\dots,p\},x\in S\}.$$This is sufficient to conclude that $x^{*}$ is the optimal solution
% of \ref{eq:Pbarv}.

% On the other hand, let $x^{*}$ be the optimal solution and $t^{*}$
% be the optimal value of \ref{eq:Pbarv}. We have $x^{*}\in\mathcal{S}$
% and $t^{*}=\max\{\dfrac{f_{j}(x^{*})-v_{j}}{\hat{d}_{j}}\mid j=1,\dots,p\}\in\Bbb R$
% so that $(x^{*},t^{*})$ belongs to the feasible domain of \ref{eq:P_1_v}.
% We now assume there exists $x\in\mathcal{S},t\in\Bbb R$ such that
% $t<t^{*}$ and $f(x)-t\hat{d}-v\leq0$. This yields that $t\geq\max\{\dfrac{f_{j}(x)-v_{j}}{\hat{d}_{j}}\mid j=1,\dots,p\}$.
% Therefore, the objective value of $(P^{2}(f(x)))$ is less than $t^{*}$.
% This contradicts the fact that $t^{*}$ is the optimal value of \ref{eq:Pbarv}.
% In other words, $(x^{*},t^{*})$ must be the optimal solution of \ref{eq:P_1_v}.


Hơn nữa, đối với \ref{eq:Pbarv}, ta tính toán dưới-gradient của $\max\{\dfrac{f_{j}(x)-v_{j}}{\hat{d}_{j}}\mid j=1,...,p\}$ với sự giúp đỡ của mệnh đề \ref{chain_rule} và bổ đề \ref{lem:2.7}. Cụ thể,
\begin{align}
    \partial \max\{\dfrac{f_{j}(x)-v_{j}}{\hat{d}_{j}}\mid j=1,...,p\} &= \text{conv} \{\partial \dfrac{f_{j}(x)-v_{j}}{\hat{d}_{j}}\mid j\in I(x) \} \\
    & = \dfrac{1}{\hat{d_j}}\text{conv} \{\partial f_j(x(t))^\mathrm{T}\dot{x}(t) \}
\end{align}

\subsection{Xác định cận dưới $\bf{\beta}$}
Khi $h(x,y)$ là một hàm giả lồi, việc tìm cận dưới
\[
\begin{array}{c}
\beta_{k}=\min\left\{ \varphi(v)\mid v\in V^{k}\right\} ,\\
\end{array}
\]
đòi hỏi việc giải bài toán
\[
\varphi(z)=\min\{h(x,y)\mid (x,y)\in\mathcal{G},f(x)\leq z\},
\]
\noindent trong dó các hàm $f_{i}$ cũng là tựa lồi và $\mathcal{G}$ là tập lồi compact khác rỗng tạo bởi các hàm tựa lồi $g$. Ta hoàn toàn có thể sử dụng phương pháp hệ động lực với các bài toán này.

\subsection{Lược đồ nhánh cận }
Phần này của đồ án trình bày thuật toán tìm ra nghiệm tối ưu của\ref{prob_QBP} and \ref{prob_OP}. Cho một sai số $\varepsilon>0$ nhỏ bất kỳ, một điểm $z^{*}\in{\rm WMin}\mathcal{Z}$ được gọi là nghiệm $\varepsilon-$tối ưu cho bài toán \ref{prob_OP} nếu tồn tại một cận trên $\alpha^{*}$ cho bài toán\ref{prob_OP} sao cho $\alpha^{*}-\varphi(z^{*})<\varepsilon(1+\vert\varphi(z^{*})\vert)$. Bất kỳ $x^{*}\in X_{WE}$ thỏa mãn $f(x^{*})\le z^{*}$ là một nghiệm tối ưu xấp xỉ cho bài toán \ref{prob_QBP}.\\

Không gian ảnh được liên tục xấp xỉ bằng việc áp dụng phương pháp nón cắt lên đa hộp đảo phía trong. Bắt đầu từ đa hộp đảo $\mathcal{P}^{0}=\left[M,M\right]$, chúng ta tiến hành thủ tục lặp xây dựng một dãy các đa hộp đảo $\left\{ \mathcal{P}^{k}\right\} $ sao cho
\[
\mathcal{P}^{0}\subset\mathcal{P}^{1}\subset\mathcal{P}^{2}\subset\dots\subset\mathcal{P}^{k}\subset\mathcal{P}^{k+1}\subset\dots\subset\mathcal{Z}^{\diamond}.
\]

% We will make use of following notations:
% \begin{itemize}
% \item $V^{k}$ is the set of all co-proper vertices which determines copolyblock
% $\mathcal{P}^{k}=\mathcal{L}\left(V^{k}\right)$.
% \item $\alpha_{k}$ is the upper bound, $\beta_{k}$ is the lower bound.
% \end{itemize}
% At the initial step with $k=0$, we have $V^{0}=\left\{ M\right\} ,\mathcal{P}^{0}=\left[M,M\right],\alpha_{0}=+\infty$.


Tại bước lặp thứ $k$, bằng việc giải bài toán $\min\left\{ \varphi\left(v\right)\mid v\in V^{k}\right\} $
cận dưới $\beta_{k}$ được tìm thấy, sau đó chúng ta xác định $v^{k}$
sao cho $\beta_{k}=\varphi\left(v^{k}\right)$. Giải bài toán $\left(P^{2}\left(v^{k}\right)\right)$,
ta tìm được một điểm hữu hiệu yếu $z^{k}=f\left(x^{k}\right)$
của $\mathcal{Z}^+$, sau đó chúng ta tìm nghiệm $y^k$ cho bài toán \ref{prob_QBP} với $x^k$ tìm được, nếu bài toán là chấp nhận được, ta sẽ nhận được nghiệm $\left(x^{k},y^{k}\right)$.
Sau đó ta so sánh $h\left(x^{k},y^{k}\right)$ với
$\alpha_{k}$ để cập nhật cận trên. Nếu $\alpha_{k}$ và $\beta_{k}$
thỏa mãn điều kiện dừng, tức là cận trên và cận dưới đủ gần, thuật toán dừng và trả về $\left(\left(x^{k},y^{k}\right),h\left(x^{k},y^{k}\right)\right)$. Ngược lại,một tập các đỉnh chính quy đảo $V^{k+1}$ được sinh ra bởi thủ tục CopolyblockCut, tập xấp xỉ trong mới cho tập ảnh $\mathcal{P}^{k+1}=\mathcal{L}\left(V^{k+1}\right)$ và cận trên của vòng lặp mới được xác định.



% Because each $f_{j}(x)$ is a pseudoconvex function and each
% $\hat{d}_{j}>0$, from \cite{Mangasarian1969}, we have that $\max\{\dfrac{f_{j}(x)-v_{j}}{\hat{d}_{j}}\},j=1,\dots,p$
% are also pseudoconvex functions. Hence, \ref{eq:Pbarv} is
% a pseudoconvex programming problem.\end{proof} 

\begin{algorithm}[H]
\SetAlgorithmName{Algorithm}{procedure}{List of Procedures}
\renewcommand{\thealgocf}{}
\let\oldnl\nl
\newcommand{\nonl}{\renewcommand{\nl}{\let\nl\oldnl}}
\caption{\textit{Solve \eqref{prob_OP}}}
\label{algo:next_time_step}
\KwIn{Bài toán \eqref{prob_OP}, sai số $\varepsilon > 0$}
\KwOut{Nghiệm tối ưu xấp xỉ cho bài toán \eqref{prob_OP}}
Giải các bài toán $(P_{i}^{m})$ và $(P_{i}^{M}),i=1,\dots,p$ để xác định hộp $[m,M]$.

Gán $\mathcal{P}^{0}\gets[M,M]$, $V^{0}\gets\{M\}$ và chọn hướng $\hat{d}\in\mathbb{R}^{p}_{+}$
(e.g., $\hat{d}=e$).

\For{$i \gets 1$ \textbf{\textup{to}} $p$}{
    $z^{i} = M-(M_{i}-m_{i})e^{i}$, giải $\varphi(z^{i})$\\
    \eIf{$\varphi(z^{i})$ có nghiệm $(x^{i},y^{i})$}
    {$\alpha^{i}=h(x^{i},y^{i})$.}
    {$\alpha^{i}=\infty$.}
}

Gán cận trên $\alpha_{0}\gets \min\{\alpha^{i}\mid i=1,2,\dots,p\}$, $k\gets0$ và $update\gets False$.\\
Gán nghiệm tối ưu tốt nhất hiện tại $x^{*}$ tương ứng với bài toán $\varphi(z^{i})$ thỏa mãn $\varphi(z^{i})=\alpha_{0}$.\\

\SetKwBlock{Begin}{\textbf{for} {$k \gets 1$ \textup{\textbf{to}} $\infty$ \textbf{do}}}{}
\SetAlgoLined
\Begin{
    \ForEach {$v\in V^{k}$}{
    Giải $\varphi(v)$\\
    \eIf{$\varphi(v)$ có nghiệm $(x^{*}, y^{*})$}
    {\eIf{$\nexists i:f_{i}(x^{*})=m_{i}$}
    {$\beta_{v}\gets \varphi(v)$.}{$V^{k} \gets V^{k}\setminus \{v\}$.}}
    {$V^{k} \gets V^{k}\setminus \{v\}$.}
    }

    Cập nhật cận dưới $\beta_{k}\gets\min\{\beta_{v}\mid v\in V^{k}\}$
    và các đỉnh tương ứng $v^{k}\in V^{k}$ sao cho $\varphi(v^{k})=\beta_{k}$.
}
\end{algorithm}
\begin{algorithm}[H]
    \LinesNumbered
    \setcounter{AlgoLine}{26}
    \SetKwBlock{Begin}{}{end}
    \Begin{
    Giải $(P^{2}(v^{k}))$ tìm nghiệm tối ưu $(x^{k},t_{k})$ và gán
    \begin{alignat*}{2}
    w^{k} &\quad \gets &&\quad v^{k}+t_{k}\hat{d};\\
    z^{k} &\quad \gets &&\quad f(x^{k}).
    \end{alignat*}
    \If{$w^{k}=z^{k}$ \textbf{\textup{and}} $w^{k}_{i}>m_{i},\forall i$}
    {Tìm $y^{k}$ thỏa mãn $(x^{k},y^{k})\in \mathcal{G}$.\\
    \eIf{$y^{k}$ được tìm thấy}{$update \gets True$.}{$update \gets False$.}
    % \eIf{$g(x^{k},y^{k})\leq 0$}
    % {$update \gets True$.}
    % {$update \gets False$.}
    }

    \If{\textit{update} \textbf{\textup{and}} $h(x^{k},y^{k})<\alpha_{k}$}{Cập nhật cận trên
    \begin{alignat*}{2}
    \alpha_{k} &\quad \gets &&\quad h(x^{k},y^{k});\\
    x^{*} &\quad \gets &&\quad x^{k}.\\
    y^{*} &\quad \gets &&\quad y^{k}.
    \end{alignat*}}
    
    \uIf{$\alpha_{k}-\beta_{k}\leq\varepsilon (1+|\beta_k|)$}{\textbf{Terminate.}}
    \Else{Xác định tập $V^{k+1}$ bằng thủ tục \textit{CopolyblockCut};
    }
    
    Xác định xấp xỉ trong mới cho tập ảnh $\mathcal{P}^{k+1} \gets \mathcal{L}(V^{k+1})$;
    
    $\alpha_{k+1} \gets \alpha_{k}$;
}
\end{algorithm}


\section{Sự hội tụ và tính đúng đắn của thuật toán}
\begin{bd}\label{lem-max_norm} Khi $k$ tiến ra vô cùng
\[
\lim_{k\rightarrow\infty}\max_{v\in V^{k}}\left\Vert w_{v}-v\right\Vert =0,
\]

trong đó $V^{k}$ là tập các đỉnh chính quy của $P^{k}$
và $w_{v}$ là điểm hữu hiệu yếu tương ứng của $\mathcal{Z}^{+}$
đạt được bằng việc giải \ref{eq:Pbarv}.

\end{bd}

\begin{cm} Xét đỉnh $v^{k}\in\mathcal{P}^{k}$ được chọn tại bước lặp tứ $k$ và $t_{k}$ giá trị tối ưu của $\left(P^{2}\left(v^{k}\right)\right)$.
Đặt $w_{v^{k}}=v^{k}+t_{k}\hat{d}$, ta có
\begin{equation}
{\rm {\rm Vol}}\left[v^{k},w_{v}^{k}\right]=\left(t_{k}\right)^{p}{\rm Vol}\left[0,\hat{d}\right].\label{eq:Vol_d}
\end{equation}

Bổ đề sẽ đúng nếu $\max_{v\in V^{k}}\left\Vert w_{v}-v\right\Vert =0$
tại mồi vài $k\ge0$. Ngược lại, tồn tại $v^{k}\in V^{k}$ sao cho
$\left\Vert w_{v^{k}}-v^{k}\right\Vert =\max_{v\in V^{k}}\left\Vert w_{v}-v\right\Vert >0$.
Ta cũng có, $\mathcal{P}^{k}\subseteq\mathcal{P}^{k+1}\setminus\left(v^{k}-{\rm int}\mathbb{R}_{+}^{p}\right)$,
do $\left[v^{k},w_{v^{k}}\right]\subseteq\mathcal{P}^{k}$ được suy ra từ định nghĩa của $w_{v^{k}}$, thể tích của $\mathcal{P}^{k}$
thỏa mãn
\begin{equation}
{\rm Vol}\mathcal{P}^{k+1}-{\rm Vol}\mathcal{P}^{k}\ge{\rm Vol}\left[v^{k},w_{v}^{k}\right].\label{eq:Vol_P^k}
\end{equation}

Kết hợp \ref{eq:Vol_d} với \ref{eq:Vol_P^k}, ta được
\[
{\rm Vol}\mathcal{P}^{k+1}-{\rm Vol}\mathcal{P}^{k}\ge\left(t_{k}\right)^{p}{\rm Vol}\left[0,\hat{d}\right].
\]

Vì vậy
\[
\sum_{i=0}^{k}\left({\rm Vol}\mathcal{P}^{i+1}-{\rm Vol}\mathcal{P}^{i}\right)\ge\left(\sum_{i=0}^{k}\left(t_{i}\right)^{p}\right){\rm Vol}\left[0,\hat{d}\right].
\]

Ta suy ra
\[
{\rm Vol}\mathcal{Y}^{\diamond}\ge{\rm Vol}\mathcal{P}^{k+1}\ge{\rm Vol}\mathcal{P}^{k+1}-{\rm Vol}\mathcal{P}^{0}\ge\left(\sum_{i=0}^{k}\left(t_{i}\right)^{p}\right){\rm Vol}\left[0,\hat{d}\right],
\]
với mọi $k\ge1$. Do đó, bằng việc cho $k\rightarrow\infty$, chuỗi dương $\sum_{i=0}^{k}\left(t_{i}\right)^{p}$ bị chặn trên bởi
${\rm Vol}\mathcal{Z}^{\diamond}/{\rm Vol}\left[0,\hat{d}\right]$,
do đó hội tụ và $\lim_{i\rightarrow\infty}t_{i}=0$.
VÌ $\hat{d}$ bị chặn, cho bất kỳ $i\ge1$, ta có
\[
\lim_{i\rightarrow\infty}\max_{v\in V^{i}}\left\Vert w_{v}-v\right\Vert =\lim_{i\rightarrow\infty}\left\Vert w_{v^{i}}-v^{i}\right\Vert =\lim_{i\rightarrow\infty}t_{i}\left\Vert \hat{d}\right\Vert =0.
\]
\end{cm}

\begin{bd}\label{lem-sol_at_border} Khi nghiệm $(\bar{x},\bar{y})$
của bài toán \ref{prob_QBP} thỏa mãn $\exists i:f_{i}\left(\bar{x}\right)=m_{i}$,
thì $\bar{x}$ có thể được xác định bằng việc giải $\varphi\left(\bar{z}\right)$ với $\bar{z}$ là nghiệm tối ưu của bài toán
\[
\min\left\{ \varphi\left(z\right)\mid z\in\left\{ M-\left(M_{j}-m_{j}\right)e^{j}\mid j=1,2,\dots,p\right\} \right\} .
\]
\end{bd}
\begin{cm} Do $(\bar{x},\bar{y})$ là nghiệm của bài toán \ref{prob_QBP}, ta có
\begin{equation}
h\left(\bar{x},\bar{y}\right)\leq\varphi\left(z\right),\quad\forall z\in\left\{ M-\left(M_{j}-m_{j}\right)e^{j}\mid j=1,2,\dots,p\right\} .\label{eq:h_phi}
\end{equation}

Xét $z^{i}=M-\left(M_{i}-m_{i}\right)e^{i}$, vì $f_{i}\left(\bar{x}\right)=m_{i}$ và $f\left(\bar{x}\right)\leq M$, we have $f\left(\bar{x}\right)\leq z^{i}$
cho nên $\bar{x},\bar{y}$ là một nghiệm chấp nhận được của bài toán
\[
\varphi\left(z^{i}\right)=\min\{h(x,y)\mid (x,y)\in \mathcal{G},f(x)\leq z^{i}\},
\]
do đó
\begin{equation}
\varphi\left(z^{i}\right)\leq h\left(\bar{x},\bar{y}\right).\label{eq:phi_h}
\end{equation}

Từ \ref{eq:h_phi} và \ref{eq:phi_h}, ta có
\[
h\left(\bar{x},\bar{y}\right)=\varphi\left(z^{i}\right).
\]

Vì lẽ đó, $(\bar{x},\bar{y})$ có thể được xác định bằng cách chọn nghiệm tối ưu tốt nhất sau khi giải $p$ bài toán $\varphi\left(z^{j}\right),j=1,2,\dots,p$.
\end{cm}

\begin{bd}\label{lem-sol_on_MinZ} Với bất kỳ $z\in\mathcal{Z}^{\diamond}$
sao cho $\varphi\left(z\right)$ có nghiệm tối ưu, nếu ta liên tiếp giải $\left(P^{2}\left(v^{k}\right)\right)$ với đỉnh xuất phát $v^{0}\equiv z$
để nhận được $p$ điểm mới $v^{k}-(v_{i}^{k}-w_{i}^{k})e^{i}$ được thêm vào $V^{k+1}$ và loại bất kỳ đỉnh $v\in V^{k+1}$ từ $V^{k+1}$ nếu $\varphi\left(v\right)$ vô nghiệm, hai phát biểu sau sẽ đúng,

i) Nếu $\nexists(\bar{x},\bar{y})\in \mathcal{G}$ sao cho $f\left(\bar{x}\right)\in{\rm Min}\mathcal{Z}\cap\left(z-\mathbb{R}_{+}^{p}\right)$,
ta sẽ nhận được $V^{k}=\emptyset$ khi $k$ tiến ra vô cùng.

ii) Nếu $\exists(\bar{x},\bar{y})\in \mathcal{G}$ sao cho $f\left(\bar{x}\right)\in{\rm Min}\mathcal{Z}\cap\left(z-\mathbb{R}_{+}^{p}\right)$,
ta có thể tách một chuỗi $\left\{ u^{k}\right\} _{k=0}^{\infty},u^{k}\in V^{k}$ với $u^{0}\equiv z$ sao cho nếu $w_{u^{k}}$ là nghiệm hữu hiệu yếu của $\mathcal{Z}^{+}$ được suy ra từ nghiệm $\left(x^{k},t_{k}\right)$
của bài toán $(P^{2}(u^{k}))$, thì khi $k$ tiến ra vô cùng
\[
\lim_{k\rightarrow\infty}\left\Vert w_{u^{k}}-f\left(\bar{x}\right)\right\Vert =0.
\]

\end{bd}

\begin{cm}
    i) Vì $\nexists(\bar{x},\bar{y})\in \mathcal{G}$ sao cho $f\left(\bar{x}\right)\in{\rm Min}\mathcal{Z}\cap\left(z-\mathbb{R}_{+}^{p}\right)$, bằng việc ký hiệu
\begin{eqnarray*}
A & = & {\rm Min}\mathcal{Z}\cap\left(z-\mathbb{R}_{+}^{p}\right),\\
B & = & \left(z-\mathbb{R}_{+}^{p}\right)\cap\left\{ f\left(x\right)\mid (x,y)\in \mathcal{G}\right\} ,
\end{eqnarray*}
ta có
\[
A\cap B=\emptyset.
\]

Vì vậy, ta có thể giả sử rằng khoảng cách infimum giữa hai tập từ là khoảng cách Hausdoff là một số $\varepsilon>0$.

Mặt khác, theo bổ đề \ref{lem-max_norm}, ta có thể chọn một số
$k$ sao cho
\[
\lim_{k\rightarrow\infty}\max_{v\in V^{k}}\left\Vert w_{v}-v\right\Vert <\varepsilon.
\]

Do đó, vì $w_{v}\in A,\forall v\in V^{k}$, không tồn tại
$v\in V^{k}$ sao cho $\exists b\in B:b\leq v$ vì khoảng cách Hausdoff $\varepsilon>0$.

Như một hệ quả, $\nexists(x,y)\in \mathcal{G}$ sao cho
$f\left(x\right)\leq v,\forall v\in V^{k}$, vì vậy $\varphi\left(v\right)$
vô nghiệm với mọi $v\in V^{k}$ kéo theo $\forall v\in V^{k}$
bị loại khỏi $V^{k}$.

ii) Ta giả sử rằng tồn tại một vài $v^{k}\in V^{k}$ thỏa mãn $v^{k}>f\left(\bar{x}\right)$ và xét $p$ điểm
\[
v^{k,i}=v^{k}-\left(v_{i}^{k}-w_{v^{k},i}\right)e^{i},i=1,2,\dots,p.
\]

Ta sẽ chứng minh rằng
\[
\exists i:v_{i}^{k,i}\geq f_{i}\left(\bar{x}\right),
\]
giải sử sai, ta có
\[
v_{i}^{k,i}<f_{i}\left(\bar{x}\right),\forall i,
\]

Mặt khác, ta cũng có
\[
w_{v^{k},i}\leq v_{i}^{k,i},\forall i,
\]
vì thế
\[
w_{v^{k},i}<f_{i}\left(\bar{x}\right),\forall i.
\]

NHư vậy, ta kết luận rằng $w_{v^{k}}<f\left(\bar{x}\right)$, nhưng điều dẫ đến mâu thuẫn bởi $w_{v^{k}},f\left(\bar{x}\right)\in{\rm WMin}\mathcal{Z}^{\diamond}$, vậy giả sử phản chứng sai và ta có
\[
\exists i:v_{i}^{k,i}\geq f_{i}\left(\bar{x}\right).
\]

Cũng bởi vì $v^{k}>f\left(\bar{x}\right)$ và $v_{j}^{k}=v_{j}^{k,i},\forall j\neq i$,
ta có
\[
\exists i:v^{k,i}\geq f\left(\bar{x}\right).
\]

Dễ thấy $v^{k,i}\in V^{k+1}$, ta có thể trích ra một dãy $\left\{ u^{k}\right\} _{k=0}^{\infty},u^{k}\in V^{k}$ với $u^{0}\equiv z$ sao cho $u^{k}>f\left(\bar{x}\right)$ bởi $u^{0}=z>f\left(\bar{x}\right)$.

Bây giờ ta sẽ chứng minh rằng, khi $k$ tiến ra vô cùng
\[
\lim_{k\rightarrow\infty}\left\Vert w_{u^{k}}-f\left(\bar{x}\right)\right\Vert =0.
\]

Vì $u^{k+1}=u^{k}-\left(u_{i}^{k}-w_{u^{k},i}\right)e^{i}$ với một vài  $i$, ta có $u^{k+1}<u^{k}$, cho nên dãy $\left\{ u^{k}\right\} _{k=0}^{\infty}$
là một dãy giảm và vì thế hội tụ do có chặn dưới $f\left(\bar{x}\right)$.

Nếu $\left\{ u^{k}\right\} _{k=0}^{\infty}$ hội tụ tại điểm $u\in{\rm Min}\mathcal{Z}\cap\left(z-\mathbb{R}_{+}^{p}\right)$ sao cho $u>f\left(\bar{x}\right)$, ta có mâu thuẫn bởi $u,f\left(\bar{x}\right)\in{\rm Min}\mathcal{Z}\cap\left(z-\mathbb{R}_{+}^{p}\right)$,
vì vậy
\[
\lim_{k\rightarrow\infty}\left\Vert u^{k}-f\left(\bar{x}\right)\right\Vert =0.
\]

Kết hợp với bổ đề \ref{lem-max_norm}, cho bất kỳ $\varepsilon>0$,
ta có thể tìm được một $u^{t}\in\left\{ u^{k}\right\} _{k=0}^{\infty}$ sao cho $\left\Vert w_{u^{t}}-u^{t}\right\Vert <\frac{\varepsilon}{2}$
and $\left\Vert u^{t}-f\left(\bar{x}\right)\right\Vert <\frac{\varepsilon}{2}$. 

Ta có bất đẳng thức tam giác
\[
\left\Vert w_{u^{t}}-f\left(\bar{x}\right)\right\Vert \leq\left\Vert w_{u^{t}}-u^{t}\right\Vert +\left\Vert u^{t}-f\left(\bar{x}\right)\right\Vert <\varepsilon.
\]

Và vì vậy ta phải có
\[
\lim_{k\rightarrow\infty}\left\Vert w_{u^{k}}-f\left(\bar{x}\right)\right\Vert =0.
\]

\end{cm}

\begin{bd}\label{lem-converge} Tại bước lặp thứ $k$, đặt
$w_{v^{k}}$ là nghiệm hữu hiệu yếu của $\mathcal{Z}^{+}$
được suy ra từ nghiệm $\left(x^{k},t_{k}\right)$ của bài toán $(P^{2}(v^{k}))$, ta xét trường hợp $\nexists i:w_{v^{k},i}=m_{i}$, thì khi $k$ tiến ta vô cùng
\[
\lim_{k\rightarrow\infty}\left\Vert \varphi(f(x^{k}))-\varphi(v^{k})\right\Vert =0.
\]
\end{bd}

\begin{cm} Do $f,g,h$ là các hàm liên tục nhận giá trị hữu hạn và $X$ là tập lồi compact và khác rỗng, khi ấy ta có $\varphi$ cũng là hàm liên tục nhận giá trị hữu hạn.

Mặt khác, từ bổ đề \ref{lem-max_norm}
\[
\lim_{k\rightarrow\infty}\left\Vert w_{v^{k}}-v^{k}\right\Vert \le\lim_{k\rightarrow\infty}\max_{v\in V^{k}}\left\Vert w_{v}-v\right\Vert =0.
\]

Vì $\nexists i:w_{v^{k},i}=m_{i}$, cho nên $w_{v^{k}}\in{\rm Min}\mathcal{Z}$ và ta có $w_{v^{k}}=v^{k}+t_{k}\hat{d}=f\left(x^{k}\right)$.

Vì $\lim_{k\rightarrow\infty}\left\Vert w_{v^{k}}-v^{k}\right\Vert =0$
và $\varphi$ là hàm liên tục, ta có
\[
\lim_{k\rightarrow\infty}\left\Vert \varphi(w_{v^{k}})-\varphi(v^{k})\right\Vert =0.
\]

Vì thế
\[
\lim_{k\rightarrow\infty}\left\Vert \varphi(f(x^{k}))-\varphi(v^{k})\right\Vert =\lim_{k\rightarrow\infty}\left\Vert \varphi(w_{v^{k}})-\varphi(v^{k})\right\Vert =0.
\]
\end{cm}

\begin{dl} Nếu bài toán \ref{prob_OP} có nghiệm tối ưu, cho bất kỳ
$\varepsilon>0$, thuật toán dừng sau hữu hạn bước lặp và trả về một nghiệm $\varepsilon-$ tối ưu cho bài toán \ref{prob_OP}.
\end{dl}

\begin{cm} Theo bổ đề \ref{lem-sol_at_border}, nếu nghiệm tối ưu toàn cục $(\bar{x},\bar{y})$ trong mệnh đề \ref{prop-QWP_Y} thỏa mãn $\exists i:f_{i}\left(\bar{x}\right)=m_{i}$ tức là $f\left(\bar{x}\right)\in{\rm WMin}\mathcal{Z}\setminus{\rm Min}\mathcal{Z}$, ta sẽ có $\bar{x}$ xác định bởi thuật toán được đề xuất.

Mặt khác, khi $f\left(\bar{x}\right)$ is on ${\rm Min}\mathcal{Z}$,
vì $V^{k}$ là tập các đỉnh của đa hộp đảo xấp xỉ trong không gian ảnh, ta phải có $v^{k}\in V^{k}$ sao cho $f\left(\bar{x}\right)\in{\rm Min}\mathcal{Z}\cap\left(v^{k}-\mathbb{R}_{+}^{p}\right)$.
Áp dụng bổ đề \ref{lem-sol_on_MinZ}.ii), tồn tại dãy các cặp
$\left(u^{t},w_{u^{t}}\right)$ sao cho
\[
\lim_{k\rightarrow\infty}\left\Vert w_{u^{k}}-f\left(\bar{x}\right)\right\Vert =0,
\]
vì vậy
\[
\lim_{k\rightarrow\infty}\left\Vert \varphi\left(w_{u^{k}}\right)-\varphi\left(f\left(\bar{x}\right)\right)\right\Vert =0,
\]
ta vì thế có thể tìm $k>0$ sao cho
\[
\left\Vert \varphi\left(w_{u^{k}}\right)-\varphi\left(f\left(\bar{x}\right)\right)\right\Vert <\frac{\varepsilon}{2}.
\]

Và nhờ có bổ đề \ref{lem-converge}, ta có
\[
\lim_{k\rightarrow\infty}\left\Vert \varphi(w_{u^{k}})-\varphi(u^{k})\right\Vert =0.
\]

Từ cách xây dựng các cận trên $\alpha_{k}$ và cận dưới $\beta_{k}$,
ta có
\[
0\le\alpha_{k}-\beta_{k}=h(x^{k},y^{k})-\varphi(v^{k})=\varphi(f(x^{k}))-\varphi(v^{k}).
\]

Trong trường hợp ta chọn $v^{k}=u^{k}$, sao cho
\[
\left\Vert \varphi\left(f\left(\bar{x}\right)\right)-\varphi(u^{k})\right\Vert <\frac{\varepsilon}{2},
\]
ta sẽ nhận được
\begin{eqnarray*}
0 & \leq & \alpha_{k}-\beta_{k}\\
 & = & \varphi(f(x^{k}))-\varphi(f(u^{k}))\\
 & = & \varphi(w_{u^{k}})-\varphi(f(u^{k}))\\
 & \leq & \Big\| \varphi\left(w_{u^{k}}\right)-\varphi\left(f\left(\bar{x}\right)\right)\Big\| +\Big\| \varphi\left(f\left(\bar{x}\right)\right)-\varphi(u^{k})\Big\| \\
 & < & \varepsilon.
\end{eqnarray*}

Hơn nữa, bổ đề \ref{lem-sol_on_MinZ}.i) chỉ ra rằng nếu ta chọn một đỉnh $v^{k}\in V^{k}$ sao cho không tồn tại nghiệm chấp nhận được của ${\rm Min}\mathcal{Z}\cap\left(v^{k}-\mathbb{R}_{+}^{p}\right)$, tập đỉnh $V^{k}$ sẽ trở thành tập rỗng sau hữu hạn bước lặp và ta có thể tiếp với các đỉnh $v^{k}\in V^{k}$ khác. Thêm vào đó, nếu ta chọn một đỉnh $v^{k}\in V^{k}$ sao cho 
\[
\nexists (x,y)\in\mathcal{G}:f\left(x\right)\in\left(v^{k}-\mathbb{R}_{+}^{p}\right),
\]
bài toán $\varphi\left(v^{k}\right)$ sẽ vô nghiệm kéo theo
$v^{k}$ bị loại khỏi $V^{k}$ và thuật toán tiếp tục với một đỉnh
$v^{k}\in V^{k}$ khác.

Như vậy, ta có thể kết luận rằng thuật toán kết thúc sau hữu hạn bước lặp và $x^{*}$ là một nghiệm $\varepsilon-$ tối ưu cho bài toán \ref{prob_OP}.
\end{cm}






\section{Thử nghiệm tính toán}

Tính hiệu quả của thuật toán sẽ được minh họa trong phần này của đồ án. Các thử nghiệm được chạy trên phần cứng ....  .Code bằng Matlab.
\begin{vd}\label{ex_benson2011}Xét bài toán \ref{prob_QBP}
\[
\begin{array}{ccl}
h(x) & = & x_{1}+x_{2}^{2}\\
f_{1}(x) & = & x_{1}^{2}+x_{2}^{2}+0.4x_{1}-4x_{2}\\
f_{2}(x) & = & max\{-0.5x_{1}-0.25x_{2}-0.2,-2x_{1}+4.6x_{2}-5.8\}
\end{array}
\]
và
\[
X=\{x\in\mathbb{R}^{2}\mid Ax\le b,x\ge0,c(x)\le0\},
\]
trong đó
\[
A=\left[\begin{array}{cc}
1.0 & -2.0\\
-1.0 & 1.0\\
2.0 & 1.0\\
2.0 & 5.0\\
-1.0 & -1.0
\end{array}\right],b=\left[\begin{array}{c}
1.0\\
1.0\\
4.0\\
10.0\\
-1.5
\end{array}\right],
\]
và
\[
c(x)=0.5(x_{1}-1)^{2}+1.4(x_{2}-0.5)^{2}-1.1.
\]

\end{vd}
\begin{table}

\caption{\label{tab:ex_benson2011}The computational result of Example \ref{ex_benson2011}}
\centering
\begin{tabular}{ccccc}
\hline 
$k$ & $v^{k}$ & $\alpha^{k}$ & $\beta^{k}$ & Gap\tabularnewline
\hline 
\hline 
1 & (0.4324, 1.0675) & 1.572121 & 1.25 & 0.322121\tabularnewline
\hline 
2 & (0.5735, 0.9264) & 1.431824 & 1.25 & 0.181824\tabularnewline
\hline 
3 & (0.6889, 0.8110) & 1.346757 & 1.25 & 0.096757\tabularnewline
\hline 
31 & (0.99654, 0.5034) & 1.250012 & 1.25 & 0.000012\tabularnewline
\hline 
32 & (0.9975, 0.5024) & 1.250006 & 1.25 & 0.000006 \tabularnewline
\hline 
\end{tabular}
\end{table}
Tại bước khỏi tạo, hộp $[m,M]=[-2.52,0.49,3.29,1.20]$,
các cận trên và dưới ban đầu $\alpha_{0}=\infty$, $\beta=\varphi(M)=1.25$. Chọn $\hat{d}=(1,1)^{T}$ và $\varepsilon=0.00001$.
Thuật toán kết thúc sau 32 bước lặp trả về $x^{*}=(0.997561, 0.502439)$ với giá trị tối ưu $h(x^{*})=1.250006$. Kết quả này tốt hơn so với thuật toán được sử dụng trong \cite{Benson2012} với $x=(0.2585,1.2415)$ và $h(x)=1.7989$ và gần với giá trị tối ưu $x^* = (1.0, 0.5)$, $h(x^*) = 1.25$ hơn. Chi tiết tính toán được cho trong bảng \ref{tab:ex_benson2011}.

\begin{vd} Xét bài toán quy hoạch phân thức tựa lồi \label{ex:benson2005}

\noindent 
\begin{align*}
    {\rm min}\  h(x) = \dfrac{2x_1 + 3x_2}{4x_1 + 5x_2 + 10}\\
\end{align*}

trong đó
\begin{align*}
    f_1(x) &= \dfrac{(3x_1+x_2)^2}{(3x_1 +x_2-1)^3}\\
    f_2(x) &= \dfrac{x_{1}^{2}-2x_{1}+x_{2}^{2}-8x_{2}}{x_{2}+1}
\end{align*}
với bài toán cấp dưới $\min (f_1(x), f_2(x))$có tập chấp nhận được $$X = \{x_{1},x_{2}\in\mathbb{R} \vert 2x_{1}+x_{2} \leq 6, 3x_{1}+x_{2}  \leq  8, x_{1}-x_{2}  \leq 1, x_{1},x_{2}  \geq 1\}$$
\end{vd}

Tại bước khỏi tạo, hộp $[m,M]=[0.19,-4.34,0.42,-3.67]$,
các cận trên và dưới ban đầu $\alpha_{0}=\infty$, $\beta=\varphi(M)=0.276170$. Chọn $\hat{d}=(1,1)^{T}$ và $\varepsilon=0.01$.
Thuật toán kết thúc sau 6 bước lặp và trả về $x^{*}=(1.020247, 1.835256)$ với giá trị tối ưu $h(x^{*})=0.302334$. Chi tiết tính toán được cho trong bảng \ref{tab:ex_benson2005}.

\begin{table}[h]
\centering{}\caption{\label{tab:ex_benson2005}The computational result of Example \ref{ex:benson2005}}
\begin{centering}
\begin{tabular}{ccccc}
\hline 
$k$ & $v^{k}$ & $\alpha^{k}$ & $\beta^{k}$ & Gap\tabularnewline
\hline 
\hline 
1 & (2.095784, 1.712646) &  0.346225 & 0.276170 & 0.070055\tabularnewline
\hline 
2 & (1.781338, 2.206333) & 0.346225 & 0.285986 & 0.070055\tabularnewline
\hline 
3 & (1.454576, 2.020253) & 0.346067 & 0.285986 & 0.060081\tabularnewline
\hline 
4 & (1.161954, 1.887563) & 0.331592 & 0.31198 & 0.030346\tabularnewline
\hline 
5 & (1.300950, 1.946424) & 0.331592 & 0.315449 & 0.019604\tabularnewline
\hline 
6 & (1.020247, 1.835256) & 0.324469 & 0.315449 & 0.009020\tabularnewline
\hline 
\end{tabular}
\par\end{centering}
\end{table}

\begin{vd} 
\noindent  \label{ex:ex3}
\begin{align*}
    \min &\ h(x) = \dfrac{3x_1 + 2x_2 + 10x_3 +11}{x_1 + x_2 +x_3 +10}\\
\end{align*}
trong đó 
\[
\begin{array}{ccl}
    f_1(x) & = & \dfrac{2x_{1}+5x_{2}+3x_{3}+10}{3x_{2}+3x_{3}+10} \\
    f_2(x) & = & \dfrac{2x_{1}+4x_{2}+10}{4x_{1}+4x_{2}+5x_{3}+10} \\
    f_3(x) & = & \dfrac{x_{1}+2x_{2}+5x_{3}+10}{x_{1}+5x_{2}+5x_{3}+10}\\
    f_4(x) & = & \dfrac{x_{1}+2x_{2}+4x_{3}+10}{5x_{2}+4x_{3}+10}
\end{array}
\]
Tập chấp nhận được của bài toán cấp dưới là tập con của $\mathbb{R}^3_+$ thỏa mãn
\begin{align*}
    2x_{1}+x_{2}+5x_{3} \leq 10, \\
    x_{1}+6x_{2}+3x_{3} \leq 10,\\ 
    5x_{1}+9x_{2}+2x_{3} \leq 10,\\ 
    9x_{1}+7x_{2}+3x_{3} \leq 10
\end{align*}
\end{vd}

Tại bước khỏi tạo, hộp $[m,M] =[1.00,0.50,1.65,0.79,1.17,1.00,3.00,1.00]$, $\hat{d}=(1, 1, 1, 1)^{T}$ và $\varepsilon=0.01$.
Các cận trên và dưới ban đầu $\alpha_{0}=\infty$, $\beta=\varphi(M)=0.604$. Thuật toán kết thúc sau 7 bước lặp và trả về nghiệm $x^{*}=(0.000000, 0.775596, 0.007336)$ với giá trị tối ưu $h(x^{*})=0.607448$. Chi tiết tính toán được cho trong bảng \ref{tab:ex3}.

\begin{table}[h]
\centering{}\caption{\label{tab:ex3}The computational result of Example \ref{ex:ex3}}
\begin{centering}
\begin{tabular}{ccccc}
\hline 
$k$ & $v^{k}$ & $\alpha^{k}$ & $\beta^{k}$ & Gap\tabularnewline
\hline 
\hline 
1 & ( 0.000000, 0.483159, 0.697862) & 0.894430 & 0.604000 & 0.29043 \tabularnewline
\hline 
2 & (0.000000, 0.783986, 0.483034) & 0.818089 & 0.603704 & 0.214385 \tabularnewline
\hline 
3 & (0.000000, 0.877863, 0.084834) & 0.648966 & 0.603704 & 0.045262\tabularnewline
\hline 
4 & (0.000000, 0.822004, 0.292454) & 0.648966 & 0.603704 & 0.045262\tabularnewline
\hline 
5 & (0.000000, 0.825505, 0.127032) &0.648966 & 0.603704 & 0.045262\tabularnewline
\hline
6 & (0.000000, 0.823124, 0.223619) & 0.648966 & 0.603704 & 0.045262\tabularnewline
\hline 
7 & (0.000000, 0.775596, 0.007336) &  0.607448 & 0.603704 & 0.003745\tabularnewline
\hline 
\end{tabular}
\par\end{centering}
\end{table}


\begin{vd} Xét bài toán trong ví dụ của \cite{example4} \label{ex:ex4}
    \[
\begin{array}{ccl}
h(x) & = & -x_{1} - 0.9\\
f_{1}(x) & = & x_1\\
f_{2}(x) & = & x_2\\
g(x) &=& x_1^2 + x_2^2 - 0.81
\end{array}
\]
\[
X = \{x \in [-1, 1] \vert x_1 + x_2 + 1 \geq 0 \}
\]
\end{vd}
Tại bước khỏi tạo, hộp $[m,M] =[-0.90,-0.90,0.00,-0.00]$, $\hat{d}=(1, 1)^{T}$ và $\varepsilon=0.01$.
Các cận trên và dưới ban đầu $\alpha_{0}=\infty$, $\beta=\varphi(M)=-1.8$.
Thuật toán kết thúc sau 17 bước lặp và trả về nghiệm tối ưu $x^{*}=(-0.893699, -0.106301)$ với giá trị tối ưu $h(x^{*})=-1.793699$ rất gần với kết quả được trình bày trong \cite{example4}. Chi tiết tính toán được cho trong bảng \ref{tab:ex4}.
\begin{table}[h]
\centering{}\caption{\label{tab:ex4}The computational result of Example \ref{ex:ex4}}
\begin{centering}
\begin{tabular}{ccccc}
\hline 
$k$ & $v^{k}$ & $\alpha^{k}$ & $\beta^{k}$ & Gap\tabularnewline
\hline 
\hline 
1 & (-0.526290843366367, -0.473709156633633) & -1.426291 & --1.800000 & 0.373709\tabularnewline
\hline 
2 & (-0.775599628831051, -0.224400371168949) & -1.675600 & -1.800000 & 0.124400\tabularnewline
\hline 
3 & (-0.893699479832796, -0.106300520167204) & -1.793699 & -1.800000 & 0.006301\tabularnewline
\hline 
\end{tabular}
\par\end{centering}
\end{table}


\begin{vd} Xét một ví dụ khác trong \cite{example4}
\label{ex:ex5}
\[
\begin{array}{ccl}
    h_1(x) & = & \left(x_1-1\right)^2+\displaystyle\sum_{i=2}^{14} x_i^2+0.25\\
     f_{1}(x) & = &\displaystyle\sum_{i=1}^{14} x_i^2\\
    f_{2}(x) & = & \left(x_1-0.5\right)^2+\sum_{i=2}^{14} x_i^2\\
\end{array}
\]

\[
    X =\{ x \in \mathbb{R}^{14} \vert -1 \leq\left(x_1, x_2, \ldots, x_{14}\right) \leq 2 \}
\]
\end{vd}
Tại bước khỏi tạo, hộp  $[m,M] =[0.00,0.00,0.25,0.25]$, $\hat{d}=(1, 1)^{T}$ and $\varepsilon=0.01$. Các cận trên và dưới ban đầu$\alpha_{0}=\infty$, $\beta=\varphi(M)=0.5$.
Thuật toán kết thúc sau 17 bước lặp và trả về nghiệm tối ưu $x^{*}=(0.5, 0,0,0,0,0,0,0,0,0,0,0,0,0)$ cùng với gia trị $h(x^{*})=0.5$ rất gần với kết quả của \cite{example4}. Chi tiết tính toán được cho trong bảng \ref{tab:ex5}.
\begin{table}[h]
\centering{}\caption{\label{tab:ex5}The computational result of Example \ref{ex:ex5}}
\begin{centering}
\begin{tabular}{cccc}
\hline 
$k$  & $\alpha^{k}$ & $\beta^{k}$ & Gap\tabularnewline
\hline 
\hline 
1  & 0.812500 & 0.500000 & 0.312500\tabularnewline
\hline 
2  & 0.566406 & 0.500000 & 0.066406\tabularnewline
\hline 
3  & 0.503922 &0.500000 & 0.003922\tabularnewline
\hline 
4  & 0.500015 & 0.500000 & 0.000015\tabularnewline
\hline 
5  & 0.500000 &0.500000 & 0.000000\tabularnewline
\hline 
\end{tabular}
\par\end{centering}
\end{table}

\begin{vd}
\noindent  \label{ex:ex6}
\begin{align*}
    \min \ h(x,y) = \max \left\{ \dfrac{3x_1 + 2x_2 +10x_3y_1 +11}{x_3 + 20} ,\dfrac{y_1}{y_2+1} \right\} 
\end{align*} 

\[
\begin{array}{ccl}
    f_1(x) & = & \dfrac{2x_{1}+5x_{2}+3x_{3}+10}{3x_{2}+3x_{3}+10} \\
    f_2(x) & = & \dfrac{2x_{1}+4x_{2}+10}{4x_{1}+4x_{2}+5x_{3}+10} \\
    f_3(x) & = & \dfrac{x_{1}+2x_{2}+5x_{3}+10}{x_{1}+5x_{2}+5x_{3}+10}\\
    f_4(x) & = & \dfrac{x_{1}+2x_{2}+4x_{3}+10}{5x_{2}+4x_{3}+10}
\end{array}
\]
\begin{align*}
    X = &\{ x \in \mathbb{R}^3_+ \vert 2x_{1}+x_{2}+5x_{3} \leq 10,\\
     &x_{1}+6x_{2}+3x_{3} \leq 10, 5x_{1}+9x_{2}+2x_{3} \leq 10, 9x_{1}+7x_{2}+3x_{3} \leq 10\} 
\end{align*}
\[
\begin{array}{ccl}
    g_1(x) &= -x_2 -x_3-2y_1 -y_2 + 2   \\
     g_2(x)&= x_2 + x_3 -5y_1 +2y_2 - 1 \\
     y \in \mathbb{R}^2_+
\end{array}
\]
\end{vd}

Tại bước khỏi tạo, hộp $[m,M] =[1.14,0.68,0.99,0.98,1.27,0.92,1.14,1.17]$, $\hat{d}=(1, 1, 1, 1)^{T}$ và $\varepsilon=0.01$. Các cận trên và dưới ban đầu $\alpha_{0}=\infty$, $\beta=\varphi(M)=0.950188$.
Thuật toán dừng sau 15 bước lặp và trả về nghiệm tối ưu $x^{*}= (0.000000, 0.914574,  0.799711)$ và $y^*$ = $(0.142857, 0.000000)$ cùng với giá trị tối ưu $h(x^{*}, y^*)= 0.959104$. Chi tiết tính toán được cho trong bảng \ref{tab:ex6}.

\begin{table}[h]
\centering{}\caption{\label{tab:ex6}The computational result of Example \ref{ex:ex6}}
\begin{centering}
\begin{tabular}{ccccc}
\hline 
$k$ & $v^{k}$ & $\alpha^{k}$ & $\beta^{k}$ & Gap\tabularnewline
\hline 
\hline 
1 & ( 0.000000, 0.422286, 1.291999,  0.142857, 0) & 1.140473 & 0.950188 & 0.190285 \tabularnewline
\hline 
2 & ( 0.000000, 0.654617, 1.059668, 0.142857, 0) & 1.054878 & 0.950188 & 0.104690 \tabularnewline
\hline 
3 & (0.000000, 0.654617,1.059668, 0.142857, 0) & 1.054878 & 0.950188 & 0.104690 \tabularnewline
\hline 
4 & (0.000000, 0.654617,1.059668, 0.142857, 0) & 1.054878 & 0.950188 & 0.104690 \tabularnewline
\hline 
5 & (0.000000, 0.773403,0.940881, 0.142857, 0) & 1.011114 & 0.950188 & 0.060926 \tabularnewline
\hline
6 & (0.000000, 0.773403, 0.940881, 0.142857, 0) & 1.011114 & 0.950188 & 0.060926 \tabularnewline
\hline
7 & (0.000000, 0.822666, 0.891619, 0.142857, 0) & 0.992965 & 0.950188 & 0.042777 \tabularnewline
\hline
8 & (0.000000, 0.874935, 0.839350, 0.142857, 0) & 0.973708 & 0.950188 & 0.023520 \tabularnewline
\hline
9 & (0.000000, 0.874935, 0.839350, 0.142857, 0) & 0.973708 & 0.950188 & 0.023520 \tabularnewline
\hline
10 & (0.000000, 0.792729, 0.921556, 0.142857, 0) & 0.973708 & 0.950188 & 0.023520 \tabularnewline
\hline
11 & (0.000000, 0.792729, 0.921556, 0.142857, 0) & 0.973708 & 0.950188 & 0.023520 \tabularnewline
\hline
12 & (0.000000, 0.858380, 0.855905, 0.142857, 0) & 0.973708 & 0.950188 & 0.023520 \tabularnewline
\hline
13 & (0.000000, 0.858380, 0.855905, 0.142857, 0) & 0.973708 & 0.950188 & 0.023520 \tabularnewline
\hline
14 & (0.000000, 0.894634, 0.819651, 0.142857, 0) & 0.966450 & 0.950188 & 0.016263 \tabularnewline
\hline
15 & (0.000000, 0.914574, 0.799711, 0.142857, 0) & 0.959104 & 0.950188 & 0.008916 \tabularnewline
\hline
\end{tabular}
\par\end{centering}
\end{table}



\chapter*{Kết luận}                         % Chương 3
Đồ án này nghiên cứu bài toán tối ưu hai cấp tựa lồi
(QBP) và đề xuất các thuật toán mới giải bài toán này.
Chương 1 đã trình bày một số khái niệm cơ sở về các lớp hàm, như
hàm tựa lồi, hàm tựa lõm, hàm tựa lồi chặt, . . . cùng
nhiều ví dụ minh họa và tính chất quan trọng. Các khái niệm về tập
chuẩn, tập chuẩn đảo, đa hộp, đa hộp đảo đã được trình bày ở Mục 1.3
cùng nhiều tính chất thú vị liên quan.

Chương 2 trình bày  thuật toán giải bài toán quy hoạch giả lồi với các hàm ràng buộc tựa lồi bằng phương pháp hệ động lực. Thuật toán này đóng vai trò trung tâm của lược đồ nhánh cận được trình bày trong Chương 3. Sự đùng đắn và hội tụ của thuật toán được trình bày chi tiết cùng với một vài ví dụ minh họa.

Chương 3 đã đề xuất mới một thuật toán xấp xỉ trong giải bài toán tối ưu hai cập tựa lồi (QBP) sử dụng đa hộp đảo trên không gian ảnh để xác định tập nghiệm hữu hiệu yếu và giải thuật nhánh cận lấy trung tâm là phương pháp hệ động lực để giải cận trên và cận dưới. Tính hội tụ của thuật toán cũng đã được chứng minh đi kèm với các ví dụ minh họa.

Mặc dù đã nỗ lực hết sức, tuy nhiên trong khoảng thời gian cho phép,
việc nghiên cứu và trình bày không thể tránh khỏi nhầm lẫn thiếu sót,
tôi mong nhận được sự giúp đỡ đóng góp của thầy cô và bạn bè để đồ
án được hoàn thiện hơn.

\printbibliography[title={Danh mục tài liệu tham khảo}]
\end{document}
